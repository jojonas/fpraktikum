\documentclass{../Misc/MontavonLaTeX/Montavon}
\usepackage{wasysym}
\usepackage{multirow}
\graphicspath {{out/}}
\heads{RWTH Aachen \\ Fortgeschrittenenpraktikum}{T06 \\ Röntgenspektroskopie}{Gruppe 20 \\ 25. Februar 2015}
\begin{document}

\title{Fortgeschrittenenpraktikum \\ \quad \\ Protokoll zur Röntgenspektroskopie }
\author{\emph{Gruppe 20} \\  Jonas Lieb, 312136 \\ Jan-Niklas Siekmann, 320781 \\ \ \\  RWTH Aachen}
\maketitle
\newpage

\tableofcontents
\newpage

\section{Einleitung}


\section{Durchführung}
\subsection{Aufbau}

\subsection{Einstellungen}

\subsubsection{Einstellung des Gains}


\section{Kalibration}
\subsection{Theorie}

\subsection{Auswertung}
\subsubsection{Rohdaten}

\subsubsection{Peakfindung}


\subsection{Erstellung der Kalibrationsgeraden}

\begin{center}
\input{out/calib.tex}
\end{center}

\subsection{Fazit}


\section{Messung der unbekannten Proben}
\subsection{Theorie}

\subsection{Auswertung}
\subsubsection{Rohdaten}

\subsubsection{Abzug der Leermessung}

\subsubsection{Glättung durch Tiefpassfilter}

\subsubsection{Peakfindung}


\subsection{Fazit}


\section{Energieauflösung}
\subsection{Theorie}

\subsection{Parameterfit}

\section{Zusammenfassung}
\subsection{Ergebnisse}

\subsection{Verringerung der Fehler}


\end{document}