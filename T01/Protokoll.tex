\documentclass{../Misc/MontavonLaTeX/Montavon}
\usepackage{wasysym}
\usepackage{multirow}
\usepackage{isotope}
\usepackage[autostyle=true,german=quotes]{csquotes}
\usepackage{mathtools}
\usepackage[biblabel]{cite}
\usepackage[font=small,labelfont=bf]{caption}

\usepackage{feynmp}
\DeclareGraphicsRule{.1}{mps}{*}{}

\newcommand{\e}[1]{\ensuremath{\times 10^{#1}}}
\newcommand{\defeq}{\vcentcolon=}
\newcommand{\eqdef}{=\vcentcolon}

\graphicspath {{out/}{bilder/}{data/}}
\heads{RWTH Aachen \\ F.-Praktikum}{T01 \\ Teilchendetektoren und Strahlenschutz}{Jonas Lieb (Gruppe 20)\\ 23. März 2015} 
\date{23. März 2015}

\newcommand{\thirdwidth}{0.32\textwidth}
\newcommand{\halfwidth}{0.48\textwidth}
\newcommand{\fullwidth}{1.0\textwidth}

\setlength\parindent{0pt}
\setlength{\parskip}\medskipamount
\begin{document}

\title{Fortgeschrittenenpraktikum \\ \quad \\ Protokoll zu den Versuchen über Teilchendetektoren und Strahlenschutz}
\author{Jonas Lieb, 312136 \\ \emph{Gruppe 20} \\ \\  RWTH Aachen}
\maketitle

%\begin{abstract}
%\end{abstract}
\newpage

\setcounter{tocdepth}{2}
\tableofcontents
\newpage

\section{Einleitung}
\subsection{Theorie}
\subsubsection{Strahlung und Strahlenschutz}
In diesem Versuch werden die drei bekannten Arten von Strahlung untersucht: $\alpha$, $\beta$ und $\gamma$-Strahlung. Es wird die Absorption der Strahlung zwischen Quelle und Detektor gemessen und mit dem Literaturwert verglichen.

Für Absorption von $\alpha$-Teilchen in Luft ($76 \unit{\%}$ Stickstoff, $23 \unit{\%}$ Sauerstoff, $\approx 1 \unit{\%}$ Argon) wird beschrieben durch
\begin{align*}
- \diff{E}{x} &= \kappa \rho \frac{Z}{A} \frac{z^2}{\beta^2} \left[\ln\left(\frac{2 m_e c^2 \gamma^2 \beta^2}{I}\right) \right] 
\end{align*}
Dabei ist $\kappa = 0.154 \unit{MeV cm^2 / g}$, $\rho = 1.293 \unit{kg/m^3}$ (Dichte von Luft)
Für die Berechnung des mittleren $Z/A$ werden zuerst die Massenanteile von Stickstoff, Sauerstoff und Argon betrachtet. 
Da bei diesen Elementen das Verhältnis zwischen Atommasse und Kernzahl in erster Näherung gleich ist (wegen $Z/A \approx 0.5$), kann für das Anzahlverhältnis das Massenverhältnis genutzt werden. 
\[
Z = 0.76 \cdot 7 + 0.23 \cot 8 + 0.01 \cdot 18 = 7.34
\]


\subsubsection{Verwendete Detektortypen}

\section{Versuche}
\subsection{Absorptionsschichtdicke des Halbleiterdetektors}
\subsubsection{Aufbau und Durchführung}
\subsubsection{Auswertung}
\subsubsection{Ergebnis}

\subsection{Absorptionsschichtdicke der Ionisationskammer}
\subsubsection{Aufbau und Durchführung}
\subsubsection{Auswertung}
\subsubsection{Ergebnis}

\subsection{$\beta$-Absorption mit dem Szintillationszähler}
\subsubsection{Aufbau und Durchführung}
\subsubsection{Auswertung}
\subsubsection{Ergebnis}

\subsection{$\gamma$-Absorption mit dem Szintillationszähler}
\subsubsection{Aufbau und Durchführung}
\subsubsection{Auswertung}
\subsubsection{Ergebnis}

\subsection{Strahlenschutz}
\subsubsection{Aufbau und Durchführung}
\subsubsection{Auswertung}
\subsubsection{Ergebnis}

\section{Zusammenfassung}

\newpage
\begin{thebibliography}{xxxx}
\bibitem{anleitung} Praktikumsanleitung: Fortgeschrittenenpraktikum fur Bachelorstudenten der Physik, Versuch T1, Teilchendetektoren und Strahlenschutz. URL: \url{http://institut2a.physik.rwth-aachen.de/de/teaching/praktikum/Anleitungen/Anleitung_WS13-14.pdf} [Stand: 01.04.2015]
\end{thebibliography}


\end{document}
