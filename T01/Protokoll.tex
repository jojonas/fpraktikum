\documentclass{../Misc/MontavonLaTeX/Montavon}
\usepackage{wasysym}
\usepackage{multirow}
\usepackage{isotope}
\usepackage[autostyle=true,german=quotes]{csquotes}
\usepackage{mathtools}
\usepackage[biblabel]{cite}
\usepackage[font=small,labelfont=bf]{caption}

\usepackage{feynmp}
\DeclareGraphicsRule{.1}{mps}{*}{}

\newcommand{\e}[1]{\ensuremath{\times 10^{#1}}}
\newcommand{\defeq}{\vcentcolon=}
\newcommand{\eqdef}{=\vcentcolon}

\graphicspath {{out/}{bilder/}{data/}}
\heads{RWTH Aachen \\ F.-Praktikum}{T01 \\ Teilchendetektoren und Strahlenschutz}{Jonas Lieb (Gruppe 20)\\ 23. März 2015} 
\date{23. März 2015}

\newcommand{\thirdwidth}{0.32\textwidth}
\newcommand{\halfwidth}{0.48\textwidth}
\newcommand{\fullwidth}{1.0\textwidth}

\setlength\parindent{0pt}
\setlength{\parskip}\medskipamount
\begin{document}

\title{Fortgeschrittenenpraktikum \\ \quad \\ Protokoll zu den Versuchen über Teilchendetektoren und Strahlenschutz}
\author{Jonas Lieb, 312136 \\ \emph{Gruppe 20} \\ \\  RWTH Aachen}
\maketitle

%\begin{abstract}
%\end{abstract}
\newpage

\setcounter{tocdepth}{2}
\tableofcontents
\newpage

\section{Einleitung}
\subsection{Theorie}
\subsubsection{Strahlung und Strahlenschutz}
In diesem Versuch werden die drei bekannten Arten von Strahlung untersucht: $\alpha$, $\beta$ und $\gamma$-Strahlung. 

Als $\alpha$-Quelle wird \isotope[226]{Ra} (Radium) verwendet. Die erwarteten Peaks sind in Tabelle \ref{tbl:radium_theo} aufgelistet. Die verwendete $\beta$-Quelle ist \isotope[90]{Sr} (Strontium), welches mit bis zu $0.549 \unit{MeV}$ in \isotope[90]{Y} zerfällt und für $\gamma$-Emission wird \isotope[137]{Cs} (Caesium) verwendet, welches Photonen der Energie $1.176 \unit{MeV}$ abstrahlt.

Es wird die Absorption der Strahlung zwischen Quelle und Detektor gemessen und mit dem Literaturwert verglichen. Der Energieverlust der geladenen Teilchen wird dabei näherungsweise durch die Bethe-Bloch Formel modelliert. Beim Versuchsteil der $\alpha$-Strahlung sind verschiedene scharfe Energiepeaks vorhanden, deren Reichweite unterschiedlich ist. Die Intensität der $\beta$-Strahlung wird durch eine Exponentialfunktion genähert. Für $\gamma$-Photonen wird eine exakte Exponentialfunktion erwartet.

Im Versuchsteil zum Strahlenschutz wird die Aktivität an verschiedenen Versuchsaufbauten gemessen und mit den gesetzlichen Grenzwerten verglichen. Außerdem wird eine Kontaminationsmessung durchgeführt, bei der festgestellt werden soll, ob verwendete Gegenstände mit radioaktivem Material kontaminiert sind.

\subsubsection{Verwendete Detektortypen}
Im Versuch werden drei verschiedene Detektortypen eingesetzt: 
\begin{itemize}
\item Halbleiterdetektor: geladene Teilchen erzeugen im pn-Übergang Elektronen-Loch-Paare und werden durch die deponierte Ladung detekiert 
\item Ionisationskammer: geladene Teilchen erzeugen in einem Gas Elektronen-Ionen-Paare, die an Elektroden deponierte Ladung wird gemessen 
\item Szintillationszähler: im Szintillationsmaterial werden Photonen freigesetzt, die durch einen Photomultiplier ebenfalls in Ladung umgewandelt werden
\end{itemize}

\section{Versuche}
\subsection{Absorptionsschichtdicke des Halbleiterdetektors}
\subsubsection{Aufbau und Durchführung}
In diesem Versuch wird das \isotope[226]{Ra}-Präparat mit dem Halbleiterdetektor untersucht. Das Präparat ist auf einem Verschiebetisch vor dem Halbleiterdetektor montiert. Die Distanz des Verschiebetisches zum Detektor wird mit einer Mikrometerschraube eingestellt und auf einem Maßband abgelesen, wobei nur die relative Änderung festgestellt werden kann, da der Nullpunkt arbiträr gewählt ist. Der Halbleiterdetektor ist mit einem Multichannelanalyzer verbunden, welcher die Messdaten über USB zum Programm \enquote{Genie 2000} überträgt, wo sie im \texttt{tka}-Format abgespeichert werden.

Im ersten Versuchsteil wird das Radium-Spektrum aufgezeichnet. Dafür wird das Präparat auf die Distanz $34.95 \unit{cm}$ eingestellt, welches den nächstmöglichen Abstand zum Detektor darstellt. 
Der Lower-Level-Diskriminator bleibt ausgeschaltet ($0$) und die Messzeit beträgt $120 \unit{s}$. Die Verstärkungseinstellungen (Gain) bleiben auf den voreingestellten Werten.

Im zweiten Teil wird die Distanz des Präparates zum Detektor schrittweise erhöht und genau dann eine Aufnahme gemacht, wenn ein Peak nach manueller Begutachtung gerade aus dem Spektrum nach links herausgefahren ist, außerdem $(1.00 \pm 0.02) \unit{mm}$ davor und dahinter (Genauigkeit aufgrund der Mikrometerschraube). Da die gemessene Energie der Restenergie der Teilchen nach Passieren der Luft entspricht, werden die Peak-Energien mit Vergrößerung der Distanz geringer und sind daher in niedrigeren Kanälen des MCAs zu finden.

Im letzten Versuchsteil wird erneut die gesamt mögliche Distanz vermessen und alle $0.5 \unit{cm}$ ein Spektrum aufgenommen.

\subsubsection{Rohdaten}

\begin{figure}[htbp]
\centering
\includegraphics[width=\fullwidth]{radium_close}
\includegraphics[width=\fullwidth]{radium_close2}
\caption{Rohdaten des ersten Versuchsteils, aufgenommen bei der Distanz $34.95 \unit{cm}$. Da auf der oberen Abbildung ein hoher Peak das Spektrum stark dominiert und zusätzlich durch den Verstärker nur die unteren Kanäle genutzt werden können, zeigt das untere Spektrum einen passenden Ausschnitt.}
\label{fig:radium_close}
\end{figure}

Die Rohdaten für den ersten Versuchsteil befinden sich in Abbildung \ref{fig:radium_close}. Da ein sehr großer dynamischer Bereich vermessen wird, ist im unteren Spektrum ein eingeschränkter Bereich gezeichnet.

\begin{figure}[htbp]
\centering
\includegraphics[width=\halfwidth]{radium_peak1}
\includegraphics[width=\halfwidth]{radium_peak2}
\includegraphics[width=\halfwidth]{radium_peak3}
\includegraphics[width=\halfwidth]{radium_peak4}
\caption{Rohdaten des zweiten Versuchsteils. Die Daten wurden auf einen Ausschnitt beschränkt, da alle Kanäle mit Kanalnummern größer als 2000 Messwerte von $< 20$ aufweisen und der höchste Peak Werte bis zu $1050000$ annimmt.}
\label{fig:radium_peaks}
\end{figure}

Die ebenso eingeschränkten Rohdaten für die Vermessung der Peak-Absorptions-Distanzen befinden sich in Abbildung \ref{fig:radium_peaks}.

\begin{figure}[htbp]
\centering
%\includegraphics[width=0.4\textwidth]{radium_distance_34_9}
\includegraphics[width=0.4\textwidth]{radium_distance_35_0}
\includegraphics[width=0.4\textwidth]{radium_distance_35_5}
\includegraphics[width=0.4\textwidth]{radium_distance_36_0}
\includegraphics[width=0.4\textwidth]{radium_distance_36_5}
\includegraphics[width=0.4\textwidth]{radium_distance_37_0}
\includegraphics[width=0.4\textwidth]{radium_distance_37_5}
\includegraphics[width=0.4\textwidth]{radium_distance_38_0}
\includegraphics[width=0.4\textwidth]{radium_distance_38_5}
\includegraphics[width=0.4\textwidth]{radium_distance_39_0}
\includegraphics[width=0.4\textwidth]{radium_distance_39_5}
\caption{Rohdaten des dritten Versuchsteils. Auch hier wurden die Daten analog zu den anderen beiden Versuchsteilen beschränkt.}
\label{fig:radium_peaks}
\end{figure}


\subsubsection{Auswertung}


\begin{table}[htbp]
\centering
\begin{tabular}{|c|c|c|}
\hline
Energie & Zerfall & Reichweite \\
\hline
%$3.792 \unit{MeV}$ & \isotope[210]{Pb} $\rightarrow$ \isotope[206]{Hg} & $2.43 \unit{cm}$ \\
$4.871 \unit{MeV}$ & \isotope[226]{Ra} $\rightarrow$ \isotope[222]{Rn} & $3.52 \unit{cm}$ \\
$5.307 \unit{MeV}$ & \isotope[210]{Po} $\rightarrow$ \isotope[206]{Pb} & $4.01 \unit{cm}$ \\
$5.590 \unit{MeV}$ & \isotope[222]{Rn} $\rightarrow$ \isotope[218]{Po} & $4.34 \unit{cm}$ \\
$6.115 \unit{MeV}$ & \isotope[218]{Pb} $\rightarrow$ \isotope[214]{Pb} & $4.99 \unit{cm}$ \\
$7.69 \unit{MeV}$ & \isotope[214]{Po} $\rightarrow$ \isotope[210]{Pb} & $7.18 \unit{cm}$ \\
\hline
\end{tabular}
\caption{Erwartete $\alpha$-Peaks und deren Reichweite in Luft bei Standardbedingungen ($25 \unit{\degree C}$, $\rho = 1.184 \unit{kg/m^3} = 1.184 \e{-3} \unit{g/cm^3}$ \cite{wiki_luftdichte}).}
\label{tbl:radium_theo}
\end{table}

\begin{figure}[htbp]
\centering
\includegraphics[width=\halfwidth]{theo}
\includegraphics[width=\halfwidth]{radium_close2}
\caption{Vergleich: Simulation verschiedener Gaußpeaks gleicher Höhe ($A = 1$) mit den Breiten $\sigma = 0.07 \sqrt{E}$ (blau) und $\sigma = 0.007 \sqrt{E}$ (grün).}
\label{fig:radium_theo}
\end{figure}

\paragraph{1. Diskussion des Spektrums.} 
Die erwarten Peakenergien und Reichweiten sind in Tabelle \ref{tbl:radium_theo} dargestellt. In Abbildung \ref{fig:radium_theo} findet sich zudem ein Vergleich zwischen einer theoretischen Vorhersage. Dabei ist die geplottete Funktion eine Summe mehrerer Gaußkurven ($f(x) = A \exp\left(-(x-\mu)^2 / 2 \sigma^2 \right)$), deren Mittelwerte $\mu$ den bekannten Energien entsprechen und deren Breite mit $\sigma = 0.07 \sqrt{E}$ geschätzt ist. Die Amplitude der Gaußkurven ist für alle Peaks gleich ($A = 1$) angenommen. Es lässt sich erkennen, dass sich bei endlicher Peakbreite die Peaks bei $5.307 \unit{MeV}$ und $5.590 \unit{MeV}$  zu einem einzelnen Peak addieren. Vermutlich ist dies auch in der Messung bei Kanal $\approx 550$ der Fall. 
Außerdem lässt sich in der Messung ein starker Hintergrund feststellen, der einen Peak bei $E \rightarrow 0$ erzeugt. Dieser Hintergrundpeak dominiert das Spektrum und musste daher vor der Darstellung entfernt werden.

Insgesamt lassen sich mit dieser Interpretation alle erwarteten Peaks erkennen. Da diese Feststellung allerdings nicht während der Versuchsdurchführung getroffen wurde, sondern erst in der Auswertung, wird er bei der Abschätzung der Reichweite ausgelassen.

Nun wird die theoretische Reichweite mit der gemessenen Reichweite verglichen. Da es schwierig ist, manuell zu schätzen, wann ein Peak links aus dem Spektrum ausgetreten ist, wird ein relativ großer Fehler von $3 \unit{mm}$ angenommen. Dieser wurde während der Durchführung geschätzt.

\paragraph{2. Bestimmung der Absorptionsschichtdicke.} 
In Tabelle \ref{tbl:radium_distanz_vergleich} werden die Literaturwerte mit der gemessenen Reichweite verglichen. 

\begin{table}[htbp]
\centering
\begin{tabular}{|c|c|c|c|}
\hline
Energie & theoretische Reichweite & gemessene Reichweite & Differenz \\
\hline
$4.871 \unit{MeV}$ & $3.52 \unit{cm}$ & $(0.3 \pm 0.3) \unit{cm}$ & $(3.2 \pm 0.3) \unit{cm} $ \\
$6.115 \unit{MeV}$ & $4.99 \unit{cm}$ & $(1.9 \pm 0.3) \unit{cm}$ & $(3.1 \pm 0.3) \unit{cm} $ \\
$7.69 \unit{MeV}$ & $7.18 \unit{cm}$ & $(4.0 \pm 0.3) \unit{cm}$ & $(3.2 \pm 0.3) \unit{cm} $ \\
\hline 
\end{tabular}
\caption{Vergleich von theoretischer mit gemessener Reichweite. Von den auf der Skala abgelesenen Werten wurde bereits der Wert der nächstmöglichen Position $39.95 \unit{cm}$ abgezogen. Außerdem wurden die Peaks bei $5.307 \unit{MeV}$ und $5.590 \unit{MeV}$ ausgelassen, da sie sich in der Messung zu einem einzelnen summierten.}
\label{tbl:radium_distanz_vergleich}
\end{table}

Daraus ergibt sich ein fehlergewichteter Mittelwert für das Luftäquivalent der Dicke des zusätzlich zum gemessenen Abstand vorhandenen absorbierenden Materials zwischen Quelle und Detektor:
\[
	x = (3.17 \pm 0.17) \unit{cm}
\]


\paragraph{3. Erstellung einer Kalibrationskurve.}
Tabelle \ref{tbl:radium_kalibration} zeigt die angenommene Peakenergien der drei vermessenen Peaks. Die Kanalnummern und ihre Fehler werden per Hand abgeschätzt. Für die Berechnung der Energie am Detektor wurden die Werte der NIST Datenbank \cite{nist_alpha} und die Näherung $\Delta E = \diff{E}{x} \cdot x$ benutzt, wobei $x = (3.17 \pm 0.17) \unit{cm}$ das im vorherigen Versuchsteil bestimmte Luftäquivalent ist.

\begin{table}[htbp]
\centering
\begin{tabular}{|c|c|c||c|}
\hline
anfängliche Energie & Energieverlust & Energie am Detektor & Kanal \\
\hline 
$4.871 \unit{MeV}$ & $0.918 \unit{MeV}$ & $3.953 \unit{MeV}$ & $240 \pm 10$ \\
$6.115 \unit{MeV}$ & $0.783 \unit{MeV}$ & $5.332 \unit{MeV}$ & $815 \pm 10$ \\
$7.69 \unit{MeV}$ & $0.67 \unit{MeV}$ & $7.02 \unit{MeV}$ & $1495 \pm 8$ \\
\hline
\end{tabular}
\caption{Energie der Teilchen am Detektor, mit einer Absorptionsschicht von $(3.17 \pm 0.17) \unit{cm}$ Luft, \emph{Stopping Power} nach \cite{nist_alpha}.}
\label{tbl:radium_kalibration}
\end{table}

\begin{figure}[htbp]
\centering
\includegraphics[width=\halfwidth]{radium_calib_fit}
\includegraphics[width=\halfwidth]{radium_calib_residual}
\caption{Anpassung der Kalibrationsgerade. Auf der horizontalen Achse befinden sich die abgelesenen Kanäle mit Fehler, auf der vertikalen Achse die Energie laut Literaturwert. Als Fehler dieser wurde die Standardabweichung der Gleichverteilung von $1 / \sqrt{12}$ für die letzte Stelle angenommen. Auf der rechten Seite ist der Residuenplot gezeigt, $\chi^2 / \textrm{ndf} = 0.8$}.
\label{fig:radium_kalibration}
\end{figure}

Eine Geradenanpassung durch die drei Werte ist in Abbildung \ref{fig:radium_kalibration} gezeigt. Die Anpassung wurde mit Migrad durchgeführt, und der Fehler auf die Parameter von dem Algorithmus geschätzt. Daraus ergibt sich folgende Kalibrationsfunktion $E(n)$, wobei $n$ die Kanalnummer ist:
\[
	E(n) = (2.447 \pm 0.025) \e{-3} \unit{MeV} \cdot n + (3.356 \pm 0.027) \unit{MeV}
\]

\paragraph{4. Alternative Methode zu Erstellung einer Kalibrationskurve. }

\begin{table}[htbp]
\centerline{\begin{minipage}{1.2\textwidth}
\begin{tabular}{|c|c|c|c|c|}
\hline
Reichweite & Messdistanz & Distanz mit Setup & Restreichw. & Restenergie \\
\hline
$(3.520 \pm 0.010) \unit{cm}$ & $(35.00 \pm 0.05) \unit{cm}$ & $(3.17 \pm 0.18) \unit{cm}$ & $(0.35 \pm 0.18) \unit{cm}$ & $(0.6 \pm 0.4) \unit{MeV}$ \\
$(4.990 \pm 0.010) \unit{cm}$ & $(35.00 \pm 0.05) \unit{cm}$ & $(3.17 \pm 0.18) \unit{cm}$ & $(1.82 \pm 0.18) \unit{cm}$ & $(3.08 \pm 0.23) \unit{MeV}$ \\
$(7.180 \pm 0.010) \unit{cm}$ & $(35.00 \pm 0.05) \unit{cm}$ & $(3.17 \pm 0.18) \unit{cm}$ & $(4.01 \pm 0.18) \unit{cm}$ & $(5.31 \pm 0.16) \unit{MeV}$ \\
$(4.990 \pm 0.010) \unit{cm}$ & $(35.50 \pm 0.05) \unit{cm}$ & $(3.67 \pm 0.18) \unit{cm}$ & $(1.32 \pm 0.18) \unit{cm}$ & $(2.41 \pm 0.27) \unit{MeV}$ \\
$(7.180 \pm 0.010) \unit{cm}$ & $(35.50 \pm 0.05) \unit{cm}$ & $(3.67 \pm 0.18) \unit{cm}$ & $(3.51 \pm 0.18) \unit{cm}$ & $(4.87 \pm 0.17) \unit{MeV}$ \\
$(4.990 \pm 0.010) \unit{cm}$ & $(36.00 \pm 0.05) \unit{cm}$ & $(4.17 \pm 0.18) \unit{cm}$ & $(0.82 \pm 0.18) \unit{cm}$ & $(1.58 \pm 0.35) \unit{MeV}$ \\
$(7.180 \pm 0.010) \unit{cm}$ & $(36.00 \pm 0.05) \unit{cm}$ & $(4.17 \pm 0.18) \unit{cm}$ & $(3.01 \pm 0.18) \unit{cm}$ & $(4.39 \pm 0.18) \unit{MeV}$ \\
$(4.990 \pm 0.010) \unit{cm}$ & $(36.50 \pm 0.05) \unit{cm}$ & $(4.67 \pm 0.18) \unit{cm}$ & $(0.32 \pm 0.18) \unit{cm}$ & $(0.5 \pm 0.4) \unit{MeV}$ \\
$(7.180 \pm 0.010) \unit{cm}$ & $(36.50 \pm 0.05) \unit{cm}$ & $(4.67 \pm 0.18) \unit{cm}$ & $(2.51 \pm 0.18) \unit{cm}$ & $(3.88 \pm 0.19) \unit{MeV}$ \\
$(7.180 \pm 0.010) \unit{cm}$ & $(37.00 \pm 0.05) \unit{cm}$ & $(5.17 \pm 0.18) \unit{cm}$ & $(2.01 \pm 0.18) \unit{cm}$ & $(3.31 \pm 0.22) \unit{MeV}$ \\
$(7.180 \pm 0.010) \unit{cm}$ & $(37.50 \pm 0.05) \unit{cm}$ & $(5.67 \pm 0.18) \unit{cm}$ & $(1.51 \pm 0.18) \unit{cm}$ & $(2.67 \pm 0.25) \unit{MeV}$ \\
$(7.180 \pm 0.010) \unit{cm}$ & $(38.00 \pm 0.05) \unit{cm}$ & $(6.17 \pm 0.18) \unit{cm}$ & $(1.01 \pm 0.18) \unit{cm}$ & $(1.92 \pm 0.31) \unit{MeV}$ \\
$(7.180 \pm 0.010) \unit{cm}$ & $(38.50 \pm 0.05) \unit{cm}$ & $(6.67 \pm 0.18) \unit{cm}$ & $(0.51 \pm 0.18) \unit{cm}$ & $(0.9 \pm 0.4) \unit{MeV}$ \\
\hline
\end{tabular}
\end{minipage}}
\end{table}

\begin{table}[htbp]
\centering
\begin{tabular}{|c|c|}
\hline
Restenergie & Kanal \\
\hline
$(0.6 \pm 0.4) \unit{MeV}$ & $220 \pm 20$ \\
$(3.08 \pm 0.23) \unit{MeV}$ & $780 \pm 20$ \\
$(5.31 \pm 0.16) \unit{MeV}$ & $1460 \pm 20$ \\
$(2.41 \pm 0.27) \unit{MeV}$ & $620 \pm 20$ \\
$(4.87 \pm 0.17) \unit{MeV}$ & $1340 \pm 20$ \\
$(1.58 \pm 0.35) \unit{MeV}$ & $450 \pm 20$ \\
$(4.39 \pm 0.18) \unit{MeV}$ & $1260 \pm 20$ \\
$(0.5 \pm 0.4) \unit{MeV}$ & $150 \pm 10$ \\
$(3.88 \pm 0.19) \unit{MeV}$ & $1170 \pm 20$ \\
$(3.31 \pm 0.22) \unit{MeV}$ & $990 \pm 20$ \\
$(2.67 \pm 0.25) \unit{MeV}$ & $820 \pm 20$ \\
$(1.92 \pm 0.31) \unit{MeV}$ & $620 \pm 20$ \\
$(0.9 \pm 0.4) \unit{MeV}$ & $390 \pm 30$ \\
\hline
\end{tabular}
\end{table}

\begin{figure}[htbp]
\includegraphics[width=\halfwidth]{radium_calib2_fit}
\includegraphics[width=\halfwidth]{radium_calib2_residual}
\end{figure}

\subsubsection{Ergebnis}

\subsection{Absorptionsschichtdicke der Ionisationskammer}
\subsubsection{Aufbau und Durchführung}
\subsubsection{Auswertung}
\subsubsection{Ergebnis}

\subsection{$\beta$-Absorption mit dem Szintillationszähler}
\subsubsection{Aufbau und Durchführung}
\subsubsection{Auswertung}
\subsubsection{Ergebnis}

\subsection{$\gamma$-Absorption mit dem Szintillationszähler}
\subsubsection{Aufbau und Durchführung}
\subsubsection{Auswertung}
\subsubsection{Ergebnis}

\subsection{Strahlenschutz}
\subsubsection{Aufbau und Durchführung}
\subsubsection{Auswertung}
\subsubsection{Ergebnis}

\section{Zusammenfassung}

\newpage
\begin{thebibliography}{xxxx}
\bibitem{anleitung} Praktikumsanleitung: Fortgeschrittenenpraktikum fur Bachelorstudenten der Physik, Versuch T1, Teilchendetektoren und Strahlenschutz. URL: \url{http://institut2a.physik.rwth-aachen.de/de/teaching/praktikum/Anleitungen/Anleitung_WS13-14.pdf} [Stand: 01.04.2015]
\bibitem{nist_alpha} National Institute of Standards and Technology: stopping-power and range tables for helium ions. URL: \url{http://physics.nist.gov/PhysRefData/Star/Text/ASTAR.html} [Stand: 02.04.2015]
\bibitem{nist_electron} National Institute of Standards and Technology: stopping-power and range tables for electrons. URL: \url{http://physics.nist.gov/PhysRefData/Star/Text/ESTAR.html} [Stand: 03.04.2015]
\bibitem{wiki_luftdichte} Wikipedia: Luftdichte. URL: \url{http://de.wikipedia.org/wiki/Luftdichte} [Stand: 02.04.2015]
\end{thebibliography}


\end{document}
