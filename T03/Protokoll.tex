\documentclass{../Misc/MontavonLaTeX/Montavon}
\usepackage{wasysym}
\usepackage{multirow}
\usepackage{isotope}
\usepackage[autostyle=true,german=quotes]{csquotes}
\usepackage{mathtools}
\usepackage[superscript,biblabel]{cite}

\usepackage{feynmp}
\DeclareGraphicsRule{.1}{mps}{*}{}


\newcommand{\defeq}{\vcentcolon=}
\newcommand{\eqdef}{=\vcentcolon}

\graphicspath {{out/}{bilder/}{data/}}
\heads{RWTH Aachen \\ Fortgeschrittenenpraktikum}{T03 \\ Winkelkorrelation}{Gruppe 20 \\ 05. März 2015} 
\date{05. März 2015}

\newcommand{\thirdwidth}{0.32\textwidth}
\newcommand{\halfwidth}{0.48\textwidth}
\newcommand{\fullwidth}{1.0\textwidth}

\setlength\parindent{0pt}
\setlength{\parskip}\medskipamount
\begin{document}

\title{Fortgeschrittenenpraktikum \\ \quad \\ Protokoll zur $\gamma\gamma$-Winkelkorrelation }
\author{\emph{Gruppe 20} \\  Jonas Lieb, 312136 \\ Jan-Niklas Siekmann, 320781 \\ \ \\  RWTH Aachen}
\maketitle
\begin{abstract}
In diesem Versuch werden durch radioaktive Zerfälle verursachte Röntenquanten beobachtet. 

Im ersten Versuchsteil werden dazu die Energiespektren der Elemente Natrium, Cobalt, Cäsium und Europium bis zu ca. $1 \unit{MeV}$ analysiert. Mithilfe dieser Daten und externer Literaturwerte wird so ein Vielkanalanalysator kalibriert und die Energieauflösung des verwendeten Detektors bestimmt.

Im zweiten Versuchsteil wird die Winkelkorrelation emittierter Röntgenquanten der Elemente Natrium und Cobalt untersucht und Aussagen über die Winkelauflösung des verwendeten Messaufbaus getroffen.
\end{abstract}
\newpage

\tableofcontents
\newpage

\section{Einleitung}

Durch radioaktiven Zerfall werden fast immer Röntenquanten frei. Diese stammen dabei entweder direkt aus der Abregung des Atomkernes oder alternativ aus Annihilation von Positronen aus $\beta+$-Zerfall mit Elektronen des Materials. Bei den Übergängen werden ein oder zwei Photonen praktisch gleichzeitig freigesetzt. 

In diesem Versuch werden diese beiden Arten der Photonenproduktion berücksichtigt und untersucht.
Die Röntgenquanten werden in Szintillatoren und Photomultipliern in elektrischen Strom umgewandelt und analysiert. Dabei wird zur Aufnahme des Spektrums im ersten Versuchsteil ein Computer verwendet, im zweiten Teil wird eine Koinzidenzschaltung aufgebaut, mit derer Hilfe die Winkelkorrelation vermessen wird.

\subsection{Theorie}
\subsubsection{Gamma-Emission}
Radioaktiver $\alpha$ und $\beta$-Zerfall hinterlässt die zerfallenen Kerne im angeregten Zustand, d.h. sie besitzen eine höhere Energie als ihre Grundzustandsenergie. In einer Kaskade begeben sich die Kerne nun vom angeregten in den Grundzustand. Um Energieerhaltung während dieser Kaskade zu gewährleisten, wird die überschüssige Energie in Form von Photonen mit $\Delta E = h \nu$ abgestrahlt.

\subsubsection{Natrium}
Die $\gamma$-Quanten von Natrium entstammen einem anderen Prozess. \isotope[22]{Na} zerfällt über $\beta+$-Zerfall zu \isotope[22]{Ne}. Dabei wird ein Positron frei, welches nach kurzer Zeit mit einem Elektron des Natriums annihiliert. Aufgrund der Impulserhaltung werden dabei zeitgleich zwei Photonen mit der Energie $m_e c^2 = 511 \unit{keV}$ frei, welche in einem Winkel von $180 \unit{\degree}$ im Ruhesystem der Annihilation emittiert werden. 

\begin{figure}[htbp]
\centering
\begin{fmffile}{annihilation}
\begin{fmfgraph*}(160,100)
\fmfleft{positron,electron}
\fmfright{gamma1,gamma2}
\fmflabel{$e$}{positron}
\fmflabel{$e$}{electron}
\fmflabel{$\gamma$}{gamma1}
\fmflabel{$\gamma$}{gamma2}
\fmf{fermion}{electron,i2,i1,positron}
\fmf{photon}{i1,gamma1}
\fmf{photon}{i2,gamma2}
\end{fmfgraph*}
\end{fmffile}
\caption{Feynman-Diagramm der Annihilation}
\end{figure}

Daher ist die in diesem Versuchsteil erwartete Korrelationsfunktion trivial: 
\[
W(\theta) = \delta(\theta - \pi)
\]

\subsubsection{Cobalt}
Bei \isotope[60]{Co} wird eine 4(2)-2(2)-0-Kaskade betrachtet. Die Energiedifferenzen dabei betragen $\gamma_1 = 1173.23 \unit{keV}$ und $\gamma_2 = 1332.51 \unit{keV}$. Daher werden dort zwei distinkte Peaks erwartet. 
Das mittlere Energieniveau dieser Kaskade weist mit $0.7 \unit{ps}$ eine extrem kurze Lebensdauer auf, weshalb die beiden emittierten Photonen praktisch zeitgleich detektiert werden.

Die räumliche Verteilung der $\gamma$-Quanten ist vorerst isotrop. Da jedoch durch das erste Photon eine Quantisierungsachse festgelegt wird, ist die Emission des zweiten Photons nicht mehr rein zufällig, sondern korreliert mit dem Winkel. 
Aus theoretischen Betrachtungen, die hier nicht weiter erläutert werden, ergibt sich die folgende Winkelkorrelation:
\[
	W(\theta) = 1 + \frac{1}{8} \cdot \cos^2(\theta) + \frac{1}{24} \cdot \cos^4(\theta)
\]

\subsection{Messgeräte und Module}
\subsubsection{Photomultiplier (PM)}
Die $\gamma$-Quanten werden mit einem \emph{Szintilationsdetektor} und einem \emph{Photomultiplier} detektiert. Passiert ein $\gamma$-Quant das Szintillationsmaterial (Thalium dotiertes Natrium-Jodid), so emittiert dieses weitere Photonen. Diese werden in den Photomultiplier geleitet, wo sie mithilfe des Photoeffektes \emph{Primärelektronen} aus einem Metall lösen. Diese werden über eine Spannung beschleunigt und lösen aus einem zweiten Metallstück weitere \emph{Sekundärelektronen}. Die Kaskade setzt sich über mehrere Stufen fort, sodass die gemessene Ladung ein Vielfaches der Ladung der Primärelektronen beträgt.

\subsubsection{Single Channel Analyser (SCA)}
Ein Single Channel Analyser ist ein Modul, welches einen analogen Input analysiert und ein digitales Signal (Spannung 0 oder 1) aussendet, wenn dort ein Peak detektiert wurde, dessen Amplitude sich in einem zuvor eingestellten Fenster befindet.

\subsubsection{Multi Channel Analyser (MCA)}
Ein Multi Channel Analyser ordnet Peaks eines analogen Signals aufgrund ihrer Amplitude in verschiedene Kanäle ein. In diesem Versuch wird das analoge Signal der Photomultiplier analysiert. Da die im Photomultiplier gemessene Ladung proportional zur deponierten Energie ist, erstellt der Multi-Channel-Analyser Energiespektra (hier aus 16384 Kanälen), die zuerst kalibriert werden müssen. 

\subsubsection{Koinzidenzeinheit}
Eine Koinzidenzeinheit dient zur Erkennung zweier gleichzeitiger Signale. 

Dafür wartet das Modul zuerst darauf, dass auf dem ersten Eingang ein logischer Puls empfangen wird. Währenddessen wird das Signal des zweiten Einganges ignoriert.
Sobald ein Puls auf dem ersten Eingang festgestellt wurde, wartet das Modul für eine fest voreingestellte \emph{Auflösungszeit} auf einen Puls auf dem zweiten Kanal. Wird dieser vernommen, so legt das Modul ein positives Signal auf seinen Ausgang. 
Während dieser zweiten Phase der Koinzidenzdetektion ignoriert das Modul jegliche Pulse auf dem ersten Kanal.

Die Einstellung der Auflösungszeit hat einen starken Einfluss auf den Verlauf der Messung. Wird sie groß gewählt, so ist die Toleranz der Zeitdifferenz zwischen den beiden Pulsen sehr groß, allerdings wird das System auch schnell gesättigt, da pro Auflösungszeitintervall maximal eine Koinzidenz festgestellt werden kann. Die Größenordnung der Auflösungzeit in diesem Versuch beträgt $50 \unit{ns}$.

\subsubsection{Counter}
Ein Counter-Modul zählt die Anzahl der eingehenden positiven Pulse und stellt sie auf einem Display dar. Außerdem beinhaltet das verwendete Modul eine Uhr, die den Counter nur für eine voreingestellte Zeit sensitiv schaltet. Dies macht präzise und vergleichbare Messungen der Ereignisraten möglich.

Der Counter wird in diesem Versuch entweder hinter den SCA oder die Koinzidenzeinheit geschaltet, um Pulse mit bestimmter Energie oder koinzidente Pulse zu messen. 

\section{Durchführung}

\subsection{Aufbau}

\begin{figure}[htbp]
\centering
\includegraphics[width=\halfwidth]{Versuchsaufbau}
\includegraphics[width=\halfwidth]{NIM-Crate}
\caption{Versuchsaufbau: links: Photomultiplier , rechts: NIM-Crate}
\label{fig:Aufbau}
\end{figure}

Der Versuchsaufbau ist in Abbildung \ref{fig:Aufbau} dargestellt. Kernstück des Aufbaus ist eine zentrale Halterung für die verschiedenen verwendeten radioaktiven Proben. Um die Halterung ist eine einstellbare Bleiabschirmung angebracht. Hinter der Abschirmung befindet sich links der fest montierte Photomultiplier PM 1. Rechts im Bild befindet sich der analoge Aufbau des Photomultipliers PM 2. Dieser ist auf einem Winkelmaß justiert, sodass der Auslenkwinkel eingestellt und abgelesen werden kann, damit das gesamte Winkelspektrum von fast $180 \unit{\degree}$ abgetastet wird. Die Maße der beiden Detektoren und ihrer Abschirmungen sind:

\begin{table}[htbp]
\centering
\begin{tabular}{|c|c|c|}
\hline
 & Photomultiplier 1 (PM1)  & Photomultiplier 2 (PM2) \\
\hline \hline
Spaltbreite & $(7.45\pm0.05) \unit{mm}$ & $(6.70\pm0.05) \unit{mm}$ \\
Höhe des Detektors & $(80.75\pm0.05) \unit{mm}$ & $(80.70\pm0.05) \unit{mm}$ \\
Abstand zur Probe & $(91.5\pm0.1) \unit{mm}$ & $(78.5\pm0.1) \unit{mm}$ \\ 
\hline
\end{tabular}
\label{tbl:Detektorwerte}
\end{table}

Von den Photomultipliern werden die Signale an die Nuclear Instrument Module (kurz NIM) weitergeleitet, die sich in einem NIM-Crate (siehe Abbildung \ref{fig:Aufbau}) befinden. Die beiden linken Module sind für die Hochspannungsversorgung der Photomultiplier verantwortlich. Bei dem dritten Modul handelt sich um ein kombiniertes Modul, das einen Verstärker und einen Single-Channel-Analyser (kurz SCA) beinhaltet. Dieses Modul ist an den ersten Photomultiplier gekoppelt. Für den zweiten Photomultiplier übernimmt das vierte Modul die Verstärkung, das fünfte Modul beinhaltet den dazugehörigen SCA. Das sechste Modul ist die Koinzidenzheit, die im zweiten Versuchsteil verwendet wird. Im siebten Modulplatz steckt ein Delay-Modul zur Signalverzögerung. Schließlich folgt der Counter, dieser zählt die Signale der SCAs bzw. der Koinzidenzheit. Im letzten Steckplatz befindet sich der Multi-Channel-Analyser (kurz MCA), der im ersten Versuchsteil benötigt wird. Dieser ist wiederum über eine USB-Schnittstelle mit einem Computer verbunden, auf dem eine Auslese- und Aufnahmesoftware der Firma \emph{Canberra} gestartet wird. 


\subsection{Verwendete Isotope}
In diesem Versuch werden mehrere verschiedene radioaktive Proben verwendet. Im Folgenden die physikalischen Daten der einzelnen Proben:

\begin{table}[htbp]
\centering
\begin{tabular}{|c|c|c||c|}
\hline 
Bezeichnung & Anfangsaktivität & Halbwertszeit & am Tag des Versuchs \\
\hline 
\isotope[22]{Na} & $(370 \pm 3) \unit{kBq}$ & $(950.5 \pm 0.4) \unit{d}$ & $(106.5 \pm 0.8) \unit{kBq}$ \\
\isotope[60]{Co} & $(740 \pm 3) \unit{kBq}$ & $(1925.3\pm0.4) \unit{d}$ & $ (662 \pm 3) \unit{kBq}$ \\
\isotope[127]{Cs} & $(37.0 \pm 0.3) \unit{kBq}$ & $(11000\pm90) \unit{d}$ & $ (36.2 \pm 0.3) \unit{kBq}$ \\
\isotope[152]{Eu} & $(37.0 \pm 0.3) \unit{kBq}$ & $(4943\pm5) \unit{d}$ & $ (29.6 \pm 0.2) \unit{kBq}$ \\
\hline 
\end{tabular}
\label{tbl:activities}
\end{table}

\subsection{Einstellung des Verstärkungsfaktors}
Zunächst müssen die Verstärkungsfaktoren eingestellt werden. Dazu wird die \isotope[152]{Eu} Quelle in den Messaufbau eingesetzt und mithilfe der groben (\emph{Coarse}) und feinen (\emph{Fine}) Einstellung die Verstärkung so eingestellt, dass die Intensitätsmaxima des Europiums das gesamte Spektrum der 16384 Kanäle abdecken. Es wird das Europium-Spekrum gewählt, da dieses die höchste zu erwartende Energie von $E = 1408.013 \unit{keV}$ enthält. Daraus resultieren folgende Einstellung für die Verstärkungsfaktoren:

\begin{table}[htbp]
\centering
\begin{tabular}{|c|c|c|} \hline Gain & PM1 & PM2 \\
\hline \hline
Coarse  & 16.00 & 30 \\
Fine & 1.50 & 6.0 \\
\hline 
\end{tabular}
\end{table}


\section{Energiespektroskopie}
\subsection{Theorie}
In diesem Versuchsteil werden die Energiespektra der Elemente Natrium, Cobalt, Cäsium und Europium vermessen. Aus den Messwerten werden in der Analyse die Position und Breite der vermessenen Peaks festgestellt.
Diese Daten werden Literaturwerten von bekannten Energiepeaks der einzelnen Proben zugeordnet. 

Dadurch kann eine Ausgleichsgerade erstellt werden, die jedem Kanal einen Energiewert zuordnet. Mit dieser Geraden und den Peakbreiten werden die Energieauflösung und die Effizienz des Detektors bestimmt. Die Energieauflösung berechnet sich an den Peaks durch $\frac{\Delta E}{E}$. Die intrinsische Effizienz ist definiert als das Verhältnis der Anzahl von im Detektor gemessenen Ereignissen zur Anzahl der theoretisch möglichen Ereignisse, die den Detektor erreichen. Es lässt sich zeigen, dass für die intrinsische Effizienz gilt:
\begin{equation}
\epsilon_{int} = \frac{4\pi \cdot r^{2} \cdot m}{F_{D} \cdot A \cdot I_{\Gamma}}
\label{eq:Effizienz}
\end{equation}

Dabei ist $r$ der Abstand zwischen dem Detektor und der Probe, $m$ die gemessene Zählrate (Ereignisse pro Sekunde), $F_D$ die effektive Detektorfläche, $A$ die Aktivität der Probe und $I_\Gamma$ das Verhältnis zwischen betrachteter Strahlungsart und gesamter $\gamma$-Abstrahlung.

\subsection{Durchführung}
Es wird der oben beschriebene Grundaufbau verwendet. Der zweite Photomultiplier wird nicht benötigt. Die einzelnen Proben werden im Messaufbau platziert und ihre Photonenemission mit dem Photomultiplier jeweils $10 \unit{min}$ lang  vermessen. Das Signal wird nach der  Verstärkung direkt an den MCA geleitet. Dieser teilt die Signale aufgrund ihrer Pulshöhe in 16384 Kanäle auf. Die gemessenen Anzahlen pro Bin werden als \texttt{.tka}-Datei abgespeichert und im Folgenden analysiert.

\subsection{Auswertung}
\subsubsection{Rohdaten}
Die Rohdaten sind als schwarze Punkte in Abbildung \ref{fig:Kalib} zu sehen. 

\begin{figure}[htbp]
\centerline{\begin{minipage}{1.2\textwidth}
\includegraphics[width=\halfwidth]{EnergiespektrumNa_tka_all}
\includegraphics[width=\halfwidth]{EnergiespektrumCo_tka_all}
\includegraphics[width=\halfwidth]{EnergiespektrumCs_tka_all}
\includegraphics[width=\halfwidth]{EnergiespektrumEu_tka_all}
\end{minipage}}
\caption{Daten der Kalibrationsmessung (inkl. Gaußkurven und theoretischer Compton-Kanten)}
\label{fig:Kalib}
\end{figure}

Als Fehler auf die Binwerte wird im Folgenden die Näherung $\sigma_N = \sqrt{N + 1}$ verwendet, die für große Anzahlen eine gute Näherung der eigentlich unsymmetrischen Poisson-Konfidenzintervalle darstellt.

\subsubsection{Peakfindung}
Zur Peakfindung wird ein Algorithmus aus mehreren Schritten verwendet:
Zunächst wird die ungefähre Position $x_0$ und Breite $\Delta x_0$ eines Peaks manuell aus dem Plot abgelesen und dem verarbeitenden Programm mitgeteilt. Dieses sucht um die angegebene Position nach einem Maximum $x_\textrm{max}$. 
Die Daten werden danach auf den Bereich $(x_\textrm{max} - \Delta x, x_\textrm{max} + \Delta x)$ beschränkt.
In den beschränkten Daten wird eine Anpassung durch die Methode der kleinsten Quadrate durchgeführt. Die anzupassende Funktion ist zusammengesetzt aus einer Gaußkurve und einem linearen Untergrund:
\[
	f(x) = A e^{-\frac{(x-\mu)^2}{2 \sigma^2}} + a_0 + a_1 x
\]
Danach wird $x_\textrm{max} = \mu$ und $\Delta x = \sigma$ gesetzt, die Daten werden erneut eingeschränkt (ausgehend von den Originaldaten) und gefittet.
Dieser Vorgang wird 5 Mal wiederholt. Die Fitergebnisse befinden sich in Tabelle \ref{tbl:Kalibration}. An den $\chi^2$-Werten ist zu erkennen, dass die Anpassung meist sinnvoll gelingt, nur im Fall des Cobalts nicht, da der lineare Untergrund aufgrund der dicht aneinander liegenden Peaks nicht erkannt werden kann.

Die Fehler auf die Ergebnisse $\sigma_{x_\textrm{max}}, \sigma_{\Delta_x}$ werden der Fitroutine entnommen. Hierfür wird der Algorithmus Migrad aus der Minuit Bibliothek verwendet.
\begin{table}[htbp]
\centering
\small
\makebox[\textwidth][c]{\input{out/calib_peaks.tex}}
\caption{Peak-Werte mit statistischen Fehlern}
\label{tbl:Kalibration}
\end{table}

\subsubsection{Erstellung der Ausgleichsgeraden}
Die gemessenen Peaks werden nun manuell den Literaturwerten zugeordnet. Die zugehörigen Energien wurden einer Zerfallsdatenbank der \enquote{International Atomic Energy Agency (IAEA)}\cite{xgamma} entnommen. Die angenommenen Energien sind der letzten Spalte von Tabelle \ref{tbl:Kalibration} zu entnehmen. 
Durch das Auftragen der zugeordneten Energie gegen den Kanal des Peaks kann eine Kalibrationsgerade (Abbildung \ref{fig:Kalibrationsgerade}) gebildet werden. Dabei werden die Fehler auf die Gauß-Peaks $\sigma_{x_\textrm{max}}$ und die in der Datenbank angegebenen Fehler für die Energie benutzt. Diese Fehler werden auf die Fehler der Parameter der Geraden fortgepflanzt.

\begin{figure}[htbp]
\centerline{\begin{minipage}{1.2\textwidth}
\includegraphics[width=\halfwidth]{calib_fit}
\includegraphics[width=\halfwidth]{calibresiduum}
\end{minipage}}
\caption{Kalibrationsgerade}
\label{fig:Kalibrationsgerade}
\end{figure}

\input{out/calib.tex}
Diese Werte werden im Folgenden verwendet werden.

\subsubsection{Energieauflösung}
Die Breite der Peaks könnte u.U. aus der begrenzten Energieauflösung des Detektors stammen. 
Die Energieauflösung eines Detektors bezeichnet die Breite der möglichen Werte, die bei einem Energiepeak gemessen werden kann.
Sie wird gewöhnlicherweise wie folgt beschrieben:
\[
\frac{\Delta E}{E} = \sqrt{a^2 + \frac{b^2}{E}}
\]
Wird diese Formel für die relative Energieauflösung umgestellt, so wird deutlich, woher die beiden Parameter $a$ und $b$ stammen:
\[
\Delta E = a E \oplus b \sqrt{E}
\]
Dabei bezeichnet $\oplus$ das quadratische Addieren der Fehler.
Der erste Term, proportional zu $E$, stammt z.B. von Kalibrationsfehlern und Inhomogenitäten des Detektors. Die Größenordnung von $a$ ist für gewöhnlich $< 1 \unit{\%}$. 
Der zweite Term stammt aus  statistischen Schwankungen. Da im Szintillator und im Photomultiplier $N$ diskrete Events erzeugt werden, die Poisson verteilt sind ($\sigma_N \propto \sqrt{N}$) und deren Anzahl direkt proportional zur Energie ($N \propto E$) ist, gilt 
\[
\sigma_{E, \textrm{stat}} \propto \sigma_N \propto \sqrt{N} \propto \sqrt{E}
\]
Für $\frac{b}{\sqrt{E}}$ wird ebenfalls ein Wert von $\approx 1 \unit{\%}$ erwartet.

Die Schätzung der Parameter erfolgt wieder durch eine Anpassung mithilfe der Methode der kleinsten Quadrate. Die anzupassende Funktion dabei lautet 
\[
	\Delta_E = \sqrt{\left(a E\right)^2 + b^2 E}
\]
Die Anpassung inklusive der Ergebnisse und ihrer Fehler sind in Abbildung \ref{fig:Energieaufloesung} und Tabelle \ref{tbl:Energieaufloesung} dargestellt.

\begin{figure}[htbp]
\centerline{\begin{minipage}{1.2\textwidth}
\includegraphics[width=\halfwidth]{energyresolution_fit}
\includegraphics[width=\halfwidth]{energyresolution_residual}
\end{minipage}}
\caption{Parameteranpassung für die Energieauflösung}
\label{fig:Energieaufloesung}
\end{figure}

\begin{table}[htbp]
\centering
\input{out/energyresolution.tex}
\caption{Anpassungsergebnisse der Energieauflösung}
\label{tbl:Energieaufloesung}
\end{table}

Das Ergebnis für den Term $a$ liegt wie erwartet bei $(0.32 \pm 0.02) \unit{\%} < 1 \unit{\%}$.
Für die Evaluation des zweiten Terms wird beispielshaft die Energie $1 \unit{MeV}$ genutzt. Dadurch ergibt sich: $\frac{b}{\sqrt{E}} = (2.295 \pm 0.003) \unit{\%}$. Dieser Ergebnis liegt ebenfalls in der vorhergesagten Größenordnung um $\approx 1 \unit{\%}$. 

Anzumerken ist bei dieser Auswertung jedoch, dass die angenommenen Fehler auf die Energie und auf die Breite viel zu gering sind und sich daher ein sehr hohes $\chi^2 / \textrm{ndf}$ von $2100$ und niedrige Fehler auf die Ergebnisse ergeben.

\subsubsection{Effizienz}
Anhand von Gleichung \ref{eq:Effizienz} kann die intrinsische Effizienz des Detektors berechnet werden. Dazu werden die Messungen der Detektorgeometrie aus Tabelle \ref{tbl:Detektorwerte} benötigt. Außerdem werden die aktuellen Aktivitäten der Quellen, die Tabelle \ref{tbl:activities} entnommen werden können, sowie die Zählrate $m$ (siehe Tabelle \ref{tbl:Zaehlraten}), die aus den Messreihen berechnet wird, benötigt.

Als Abstand zur Quelle wird in diesem Fall jedoch \emph{nicht} der in Tabelle \ref{tbl:Detektorwerte} angegebene Wert benutzt, sondern der Radius $r = (1.5 \pm 0.5) \unit{cm}$ angenommen. Dies folgt der Überlegung, dass der Detektor effektiv alle Photonen misst, die die Bleiabschirmung passieren. Der Abstand der Metallabschirmung zur Probe wurde dabei im Nachhinein abgeschätzt.

Die Ergebnisse der Effizienzberechnung befinden sich in der Tabelle \ref{tbl:Effizienzen}. Es ist zu erkennen, dass die bestimmten Werte einen sehr großen Fehler besitzen, der genauere Aussagen über den Aufbau zunichte macht. Ursache dafür ist vermutlich die ungenau abgeschätzte Distanz der Probe zur Metallblende.

\begin{table}[htbp]
\centering
\input{out/rates.tex}
\caption{Zählraten $m$ der einzelnen Messungen}
\label{tbl:Zaehlraten}
\end{table}

\begin{table}[htbp]
\centering
\begin{tabular}{|l|l|}
\hline
El. & Effizienz  \\
\hline\hline
Na & $(25 \pm 16) \unit{\%}$ \\
Co & $(8 \pm 6) \unit{\%}$ \\
Cs & $(8 \pm 6) \unit{\%}$ \\
Eu & $(26 \pm 18) \unit{\%}$ \\
\hline
\end{tabular}
\caption{Berechnete Detektoreffizienz bei den Messungen der einzelnen Elemente.}
\label{tbl:Effizienzen}
\end{table}

\subsection{Ergebnis}
In diesem Versuchsteil wurden die Energiespektra von Natrium, Cobalt, Cäsium und Europium aufgenommen. Den Messwerten wurden Energien aus Literaturwerten zugewiesen, sodass sich eine Kalibrierung vn $98.8 \unit{eV/Kanal}$ mit einem Offset von $-19 \unit{keV}$ ergab.
Die beiden Komponenten der Energieauflösung wurden bestimmt, sie führen zu einer Auflösung von $\approx 2.3 \unit{\%}$. 
Außerdem wurde die intrinsische Detektoreffizienz bestimmt, sie beträgt zwischen $5$ und $30 \unit{\%}$.


\section{Koinzidenzmessung}
In diesem Versuchsabschnitt werden unterschiedliche $\gamma\gamma$-Koinzidenzen gemessen. Im ersten Versuch wird der Annihilationspeak von \isotope[22]{Na} untersucht, bei dem beide $\gamma$-Quanten in einem Winkel von $180 \degree$ zueinander emittiert werden. 
Im zweiten Teil wird die Winkelkorrelation der zwei $\gamma$-Quanten gemessen, die in der Kaskade von \isotope[60]{Co} entstehen. 

\subsection{Annihilationspeak von \isotope[22]{Na}}
\subsubsection{Theorie}
Das durch den $\beta^{+}$ Zerfall von \isotope[22]{Na} entstehende Positron annihiliert mit einem Elektron des Natriums. Die beiden dabei entstehenden Photonen der Energie $511 \unit{keV}$ werden in einem Winkel von $180 \unit{\degree}$ zueinander emittiert.

\subsubsection{Durchführung}
Es wird der Aufbau aus Abbildung \ref{fig:Aufbau} verwendet. Dieses Mal werden beide Photomultiplier benutzt. Außerdem werden beide SCA-Module benötigt, deren Ausgänge mit der Koinzidenzeinheit verbunden werden. Die Koinzidenzen werden schließlich mit einem Counter am Ausgang der Koinzidenzeinheit gezählt. Desweiteren wird im Lauf des Versuchs das Delay-Modul benötigt. 
Zunächst wird auf der Winkelskala ein Winkel von $0 \unit{\degree}$ eingestellt, um die Grundeinstellungen vorzunehmen. 

Im ersten Schritt werden beide SCA-Module eingestellt. Dazu wird zuerst der Verstärker des Photomultipliers 1 mit dem dazugehörigen SCA verbunden, und dieser wiederum direkt an den Counter angeschlossen. Im SCA wird ein festes Fenster von $0.2 \unit{V}$ eingestellt. In $0.1 \unit{V}$-Schritten wird das Spektrum von $0$ bis $10 \unit{V}$ abgetastet, und dabei in jedem Schritt die Anzahl der Events in $5 \unit{s}$ gezählt. Dieser gesamte Vorgang wird ebenfalls für PM 2 durchgeführt. 

\begin{figure}[htbp]
\centerline{\begin{minipage}{1.2\textwidth}
\includegraphics[width=\halfwidth]{window_0}
\includegraphics[width=\halfwidth]{window_1}
\end{minipage}}
\caption{SCA-Fenstersuche für Natrium}
\label{fig:SCA_Window_Na}
\end{figure}

Für eine möglichst hohe Zählrate werden im weiteren Verlauf des Versuches die Einstellungen aus Tabelle \ref{tbl:SCA_Window_Na} benutzt, welche nach manueller Abschätzung so gewählt wurden, dass der gesamte Photopeak im gewählten Fenster liegt.

\begin{table}[htpb]
\centering
\begin{tabular}{|c|c|c|}
\hline 
 & SCA 1 & SCA 2 \\
\hline \hline
$E$ & $2.8 \unit{V}$ & $3.2 \unit{V}$ \\
$\Delta E$ & $0.8 \unit{V}$ & $0.6 \unit{V}$ \\
\hline
\end{tabular}
\caption{Einstellungen der SCAs für die Vermessung von Natrium}
\label{tbl:SCA_Window_Na}
\end{table}

Da nicht davon ausgegangen werden kann, dass beide Schaltungen die Signale in gleicher Zeit verarbeiten, muss im folgenden Schritt die optimale Verzögerung der Verarbeitungskanäle bestimmt werden.
Dazu wird das Delay-Modul nacheinander in je einen der Schaltzweige eingefügt und dort der gesamte Bereich der möglichen Verzögerungen abgetastet. Hierbei werden beide SCAs bereits mit der Koinzidenzeinheit verbunden. Für jede eingestellte Verzögerung wird $10 \unit{s}$ lang die Anzahl der Ereignisse an der Koinzidenzeinheit gezählt.
Die Ergebnisse dieser Messung sind in Abbildung \ref{fig:Delay} zu sehen. Zur Bestimmung der optimalen Verzögerung wurde in der Analyse eine Gaußkurve angepasst, während des Versuches wurde diese Verzögerung jedoch manuell mit $-45 \unit{ns}$ abgeschätzt, daher wird dieser Wert im Folgenden benutzt. Der negative Wert bedeutet dabei, dass PM 1 verzögert wurde.

\begin{figure}[htbp]
\centerline{\begin{minipage}{1.2\textwidth}
\includegraphics[width=\halfwidth]{delay_fit}
\includegraphics[width=\halfwidth]{delay_residual}
\end{minipage}}
\caption{Auswirkung der Verzögerung auf gemessene Anzahl der Koinzidenzen}
\label{fig:Delay}
\end{figure}

Da nun die optimalen Einstellungen bekannt sind, wird mit der eigentlichen Messung der Winkelkorrelation begonnen: Der Ausgang des Single-Channel-Analysers 1, der auf die Fenstereinstellung aus Tabelle \ref{tbl:SCA_Window_Na} eingestellt ist, wird mit einem Delay von $45 \unit{ns}$ an die Koinzidenzeinheit angeschlossen. Der Ausgang von SCA 2 wird direkt mit dem zweiten Eingang der Koinzidenzeinheit verbunden.
In $2 \unit{\degree}$-Schritten wird nun das Spektrum von $-10$ bis $10 \unit{\degree}$ vermessen. Dabei wird bei jeder Einstellung eine Messung von $120 \unit{s}$ vorgenommen, während dieser die Anzahl der Ereignisse gezählt wird. 

Der Winkel wird mithilfe der Winkelskala auf ca. $0.3 \unit{\degree}$ Genauigkeit eingestellt.

\subsubsection{Messdaten}
Die Messergebnisse der Winkelkorrelation von $-10$ bis $10 \unit{\degree}$ sind in Abbildung \ref{fig:NA22} zu sehen. Angepasst ist eine Gaußverteilung ohne konstanten Offset, weil die zufällige Koinzidenz vernachlässigbar klein ist. Als Fehler auf die Anzahl der Ereignisse wurde erneut $\sigma_N = \sqrt{N+1}$ angenommen, der Fehler auf den Winkel beträgt $0.3 \unit{\degree}$.

\begin{figure}[htbp]
\centerline{\begin{minipage}{1.2\textwidth}
\includegraphics[width=\halfwidth]{coincidence_na_2_fit}
\includegraphics[width=\halfwidth]{coincidence_na_2_residual}
\end{minipage}}
\caption{Annihilationspeak von \isotope[22]{Na}}
\label{fig:NA22}
\end{figure}

\subsubsection{Ergebnis}
Die Mitte des angepassten Peaks liegt nicht wie erwartet bei einem Winkel von $0 \unit{\degree}$. Stattdessen scheint die Winkelskala einen Offset von \[
	\mu = (1.05 \pm 0.13) \unit{\degree}
\] aufzuweisen. Um diesen Mittelwert ist die Anzahl der Ereignisse gaußverteilt, mit einer Standardabweichung von \[
	\sigma = (10.12 \pm 0.22) \unit{\degree}
\]Die Anpassung liegt mit $\chi^2 / \mathrm{ndf} = 0.5$ sehr gut im Rahmen der Fehler. 

Da aufgrund technischer Schwierigkeiten der Messaufbau im zweiten Versuchsaufbau signifikant verändert wurde, werden diese Fehler \emph{nicht} als systematischer Fehler in die weiteren Messungen einfließen.

\subsection{Winkelkorrelation von \isotope[60]{Co}}
\subsubsection{Theorie}
Im Gegensatz zur Messung des Annihilationspeaks von \isotope[22]{Na} werden bei der Messung der Winkelkorrelation von \isotope[60]{Co} die $\gamma$-Quanten zweier verschiedener Peaks aus dem Cobalt-Spektrum miteinander in Verbindung gebracht. Diese Peaks werden bei $1173 \unit{keV}$ und $1332 \unit{keV}$ erwartet, die gemessenen Werte sollten proportional sein zu
\[
	W(\theta) = 1 + \frac{1}{8} \cdot \cos^2(\theta) + \frac{1}{24} \cdot \cos^4(\theta)
\]

\subsubsection{Durchführung}
Die Durchführung ist analog zur Durchführung des vorherigen Versuchsteils: zuerst werden die optimalen SCA-Fenster ermittelt. Dafür wird erneut das Spektrum von $0$ bis $10 \unit{V}$ untersucht, diesmal in $0.2 \unit{V}$-Schritten, bei einer Fenstergröße von ebenfalls $0.2 \unit{V}$ und einer Messdauer von je $5 \unit{s}$.

Die Rohdaten der Messung finden sich in Abbildung \ref{fig:SCA_Window_Co}. Die beiden erwarteten Peaks sind deutlich zu erkennen.

\begin{figure}[htbp]
\centerline{\begin{minipage}{1.2\textwidth}
\includegraphics[width=\halfwidth]{window_2}
\includegraphics[width=\halfwidth]{window_3}
\end{minipage}}
\caption{SCA-Fenstersuche für Cobalt}
\label{fig:SCA_Window_Co}
\end{figure}

Anhand dieser Ergebnisse werden die beiden SCAs auf Energien der verschiedenen Peaks eingestellt (siehe Tabelle \ref{tbl:SCA_Window_Co}).

\begin{table}[htbp]
\centering
\begin{tabular}{|c|c|c|}
\hline 
 & SCA 1 & SCA 2 \\
\hline \hline
$E$ & $7.0 \unit{V}$ & $8.6 \unit{V}$ \\
$\Delta E$ & $0.8 \unit{V}$ & $1.0 \unit{V}$ \\
\hline
\end{tabular}
\caption{Einstellungen der SCAs für die Vermessung von Cobalt}
\label{tbl:SCA_Window_Co}
\end{table}

In diesem Versuchsteil wird kein Delay verwendet, da die Koinzidenzrate so gering ist,  dass der Vorversuch zur Einstellung des Delays zu viel Zeit in Anspruch genommen hätte.

Die Auflösungszeit der Koinzidenzeinheit wurde für diese Versuchsreihe auf $100 \unit{ns}$ gesetzt. 

Im Hauptteil des Versuches wird nun der Winkelbereich von $-75 \unit{\degree}$ bis $75 \unit{\degree}$ in $10 \unit{\degree}$-Schritten vermessen. Dabei beträgt der Fehler auf den eingestellten Winkel erneut $0.3 \unit{\degree}$. Die Messzeit jeder Winkeleinstellung beträgt $600 \unit{s}$.

\subsubsection{Messdaten}
Die Abhängigkeit der Anzahl der Ereignisse vom Auslenkwinkel ist dargestellt auf der linken Seite der Abbildung \ref{fig:Co60}.

\begin{figure}[htbp]
\centerline{\begin{minipage}{1.2\textwidth}
\includegraphics[width=\halfwidth]{coincidence_co_fit}
\includegraphics[width=\halfwidth]{coincidence_co_residual}
\end{minipage}}
\caption{Winkelkorrelation von \isotope[60]{Co}}
\label{fig:Co60}
\end{figure}

Es ist zu erkennen, dass die Werte sehr stark statistisch streuen, um einen Mittelwert von ca. 230.

\subsubsection{Auswertung}
An diese Messwerte wird die erwartete Funktion 
\[
	W(\theta) = 1 + a_2 \cdot \cos^2(\theta) + a_4 \cdot \cos^4(\theta)
\]
angepasst. Dabei werden die Werte $a_2 = \frac{1}{8} = 0.1250$ und $a_4 = \frac{1}{24} = 0.0417$ erwartet.
Die Ergebnisse sind 
\[
	a_2 = (-0.10 \pm 0.21) \enskip \textrm{und} \enskip a_4 = (0.12 \pm 0.20)
\]
Innerhalb ihrer sehr großen Fehler stimmen diese Werte zwar mit den theoretischen Voraussagen überein, und $\chi^2 / \textrm{ndf} = 16 / 14 = 1.14$ ließe vermuten, dass die Anpassung sinnvoll ist, jedoch kann an die selben Daten ebenso eine konstante Funktion 
\[ 
	W(\theta) = c
\] angepasst werden, mit dem Ergebnis $c = (233.8 \pm 3.7)$, $\chi^2 / \textrm{ndf} = 17 / 16 = 1.06$. 
Daher kann diese Messung das erwartete Ergebnis nicht bestätigen, die Messdaten weisen eher die Charakteristika einer Gleichverteilung auf. Dies spricht für eine isotrope Winkelverteilung.

\subsection{Ergebnis}
Die Messung der Korrelationsfunktion des Natrium-Isotops konnte die theoretische Vorhersage bestätigen. Es zeigte sich ein Gauß-Peak, der um ca. $0 \unit{\degree}$ zentriert ist.
Die theoretischen Vorhersagen der Winkelkorrelationsmessung von Cobalt konnten wiederum nicht bestätigt werden. Die Messwerte streuen stark um den Mittelwert von 234 in 10 Minuten und sind nahezu gleichverteilt.

\subsection{Fehlerbetrachtung}
Vermutlich handelt es sich bei den gemessenen Ereignissen ausschließlich um zufällige Koinzidenzen. Bei diesen wird erwartet, dass sie im Winkel gleichverteilt sind. Um diese Aussage zu überprüfen, kann die Anzahl der erwarteten zufälligen Koinzidenzen berechnet werden.
Dafür bietet sich folgende Formel an:
\[
	m^\mathrm{coinc}_\mathrm{rand} = T_A \cdot m_1 \cdot m_2
\]
Um die Raten $m_1$ und $m_2$ abzuschätzen, wird noch einmal auf Abbildung \ref{fig:SCA_Window_Co} verwiesen. Innerhalb der gewählten Fenster befinden sich für PM 1 17783 Ereignisse und für PM 2 19311 Ereignisse in je $5 \unit{s}$. Daher ist 
\[
m_1 = \frac{(17783 \pm 133)}{5 \unit{s}} = (3557 \pm 27) \unit{1/s} \enskip \mathrm{und} \enskip m_2 = \frac{(19311 \pm 139)}{5 \unit{s}} = (3862 \pm 28) \unit{1/s} 
\]
Mit einer Auflösungszeit von $T_A = 100 \unit{ns}$ und einer Messdauer von $T = 600 \unit{s}$ ergibt dies eine erwartete zufällige Koinzidenz von 
\begin{align*}
	N^\mathrm{coinc}_\mathrm{rand} &= T \cdot m^\mathrm{coinc}_\mathrm{rand} \\
	&= 600 \unit{s} \cdot 100 \unit{ns} \cdot (3557 \pm 27) \unit{1/s} \cdot (3862 \pm 28) \unit{1/s} \\
	&= 600 \unit{s} \cdot (1.374 \pm 0.014) \unit{1/s} \\
	&= 824 \pm 9
\end{align*}
Diese Zahl ist bereits sehr viel größer als die ca. 233 tatsächlich gemessenen zufälligen Koinzidenzen pro Bin. Dies spricht für zusätzliche Fehler in der Koinzidenzschaltung, da (statistisch) mindestens $824$ Ereignisse pro Bin hätten gemessen werden sollen.

\section{Zusammenfassung}
In diesem Versuch wurde $\gamma$-Emission vierer Elemente (Natrium, Cobalt, Cäsium und Europium) untersucht, die Energiespektra aufgenommen und ausgewertet. Dabei ist es gelungen, eine Energiekalibration der MCA-Kanäle vorzunehmen. 
Diese Kalibration wurde daraufhin dazu benutzt, die Energieauflösung des Detektors zu bestimmen, die sich wie erwartet im unteren Prozentbereich befand. 
Im Hauptteil des Versuches wurden die Winkelkorrelationen der Photonen von Natrium und Cobalt betrachtet. Bei Natrium zeigte sich, wie erwartet, eine Gaußverteilung bei einer Winkeldifferenz von $180 \unit{\degree}$. Im Fall des Cobalts konnte die Messung die Erwartungen nicht bestätigen. Es wurden deutlich weniger Ereignisse gemessen als erwartet, in der Auswertung zeigte sich zudem, dass die vorhergesagte Winkelkorrelation nicht bestimmt werden konnte.

\newpage
\begin{thebibliography}{xxxx}
\bibitem{anleitung} Praktikumsanleitung: Versuch T03, $\gamma \gamma$-Winkelkorrelation. URL: \url{http://institut2a.physik.rwth-aachen.de/de/teaching/praktikum/Anleitungen/T03.pdf} [Stand: 16.03.2015]
\bibitem{xgamma} International Atomic Energy Agency: X-Ray and Gamma-ray Decay Data Standards for Detector Calibration and Other Applications. URL: \url{https://www-nds.iaea.org/xgamma_standards/genergies1.htm}
\end{thebibliography}


\end{document}
