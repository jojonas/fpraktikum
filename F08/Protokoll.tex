\documentclass{../Misc/MontavonLaTeX/Montavon}
\usepackage{wasysym}
\usepackage{multirow}
\usepackage{isotope}
\usepackage[autostyle=true,german=quotes]{csquotes}
\usepackage{mathtools}
\usepackage[superscript,biblabel]{cite}

\usepackage{feynmp}
\DeclareGraphicsRule{.1}{mps}{*}{}


\newcommand{\defeq}{\vcentcolon=}
\newcommand{\eqdef}{=\vcentcolon}

\graphicspath {{out/}{bilder/}{data/}}
\heads{RWTH Aachen \\ Fortgeschrittenenpraktikum}{F08 \\ Neodym-YAG-Laser}{Jonas Lieb (Gruppe 20)\\ 12. März 2015} 
\date{12. März 2015}

\newcommand{\thirdwidth}{0.32\textwidth}
\newcommand{\halfwidth}{0.48\textwidth}
\newcommand{\fullwidth}{1.0\textwidth}

\setlength\parindent{0pt}
\setlength{\parskip}\medskipamount
\begin{document}

\title{Fortgeschrittenenpraktikum \\ \quad \\ Protokoll zu den Versuchen des Neodym-YAG-Lasers}
\author{Jonas Lieb, 312136 \\ \emph{Gruppe 20} \\ \\  RWTH Aachen}
\maketitle

\begin{abstract}
...
\end{abstract}
\newpage

\tableofcontents
\newpage

\section{Einleitung}
In diesem Versuch wird ein Neodym-YAG-Laser zusammengesetzt und charakterisiert. Nd-YAG-Laser haben eine Wellenlänge von $\lambda = 1064 \unit{nm}$, diese liegt im für das menschliche Auge unsichtbaren Infrarotbereich. 
Gepumpt wird dieser Laser mit einer Halbleiter-Laserdiode. Auch diese wird zunächst auf ihre Temperaturabhängigkeit und Ausgangsleistung untersucht.
Im Anschluss wird ein KTP-Kristall in den Resonator des Nd-YAG-Lasers eingebaut, sodass eine Frequenzverdopplung auf $532 \unit{nm}$ stattfindet. Das resultierende Licht ist für das menschliche Auge grün sichtbar.
Im letzten Versuchsteil wird der Laser in den Pulsbetrieb versetzt. Dafür wird statt des KTP-Kristalls eine Pockelszelle in den Resonator verbaut, die von einem Frequenzgenerator getrieben wird und zur aktiven Gütemodulation dient. Die Peakleistung dieser Pockelszelle wird mithilfe eines Photosensors bestimmt.

\subsection{Theorie}
\subsubsection{Laser-Grundlagen}
Das Wort \enquote{Laser} bedeutet \emph{Light Amplification by Stimulated Emission Radiation}, also \enquote{Lichtverstärkung durch stimulierte Emission von Strahlung}. Stimulierte Emission ist die wichtigste Wechselwirkung von Photonen mit Materie in einem Laser. Dabei wechselwirkt ein einfallendes Photon mit einem Elektron auf einem höheren Energieniveau $E_2$, sodass dieses auf ein niedrigeres Energieniveau $E_1$ herabfällt und ein weiteres Photon aussendet, das dieselbe Wellenlänge und Richtung, sowie eine feste Phasenbeziehung zu dem einfallenden Photon besitzt. So existieren nach diesem Prozess zwei statt einem Photon. Die Frequenz $\nu$ des einfallenden und emittierten Photons ist dabei durch die Energiedifferenz der beiden Niveaus festgelegt: $h \nu = \Delta E = E_2 - E_1$.

Neben stimulierter Emission existieren außerdem zwei weitere Prozesse: Spontane Emission und Absorption. Bei spontaner Emission wird der Übergang des Elektrons vom höheren ins niedrigere Energieniveau nicht durch Strahlung verursacht, sondern geschieht spontan. Dabei wird ebenfalls ein Photon emittiert, jedoch in eine beliebige Richtung und mit beliebiger Phase. Absorption bezeichnet den Wechselwirkungsprozess, in dem ein Photon von einem Elektron auf einem niedrigeren Niveau eingefangen wird und so das Elektron auf ein hohes Niveau bringt. Dabei wird das Photon vollständig absorbiert.

\section{Charakterisierung des Diodenlasers}

\section{Charakterisierung des Nd-YAG-Lasers}

\section{Frequenzverdopplung mit KTP}

\section{Aktive Gütemodulation}

\section{Zusammenfassung}

\newpage
\begin{thebibliography}{xxxx}
\bibitem{anleitung} Praktikumsanleitung:
Versuch F08, Neodym-YAG-Laser. URL: \url{http://institut2a.physik.rwth-aachen.de/de/teaching/praktikum/Anleitungen/B_FK08-Nd-YAG_Teil1_12-03-2012.pdf}, \url{http://institut2a.physik.rwth-aachen.de/de/teaching/praktikum/Anleitungen/B_FK08-Nd-YAG-Teil2_18-01-2012.pdf} [Stand: 20.03.2015]
\end{thebibliography}


\end{document}
