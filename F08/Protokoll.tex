\documentclass{../Misc/MontavonLaTeX/Montavon}
\usepackage{wasysym}
\usepackage{multirow}
\usepackage{isotope}
\usepackage[autostyle=true,german=quotes]{csquotes}
\usepackage{mathtools}
\usepackage[biblabel]{cite}
\usepackage[font=small,labelfont=bf]{caption}

\usepackage{feynmp}
\DeclareGraphicsRule{.1}{mps}{*}{}


\newcommand{\defeq}{\vcentcolon=}
\newcommand{\eqdef}{=\vcentcolon}

\graphicspath {{out/}{bilder/}{data/}}
\heads{RWTH Aachen \\ Fortgeschrittenenpraktikum}{F08 \\ Neodym-YAG-Laser}{Jonas Lieb (Gruppe 20)\\ 12. März 2015} 
\date{12. März 2015}

\newcommand{\thirdwidth}{0.32\textwidth}
\newcommand{\halfwidth}{0.48\textwidth}
\newcommand{\fullwidth}{1.0\textwidth}

\setlength\parindent{0pt}
\setlength{\parskip}\medskipamount
\begin{document}

\title{Fortgeschrittenenpraktikum \\ \quad \\ Protokoll zu den Versuchen des Neodym-YAG-Lasers}
\author{Jonas Lieb, 312136 \\ \emph{Gruppe 20} \\ \\  RWTH Aachen}
\maketitle

%\begin{abstract}
%\end{abstract}
\newpage

\setcounter{tocdepth}{2}
\tableofcontents
\newpage

\section{Einleitung}
In diesem Versuch wird ein Nd-YAG-Laser zusammengesetzt und charakterisiert. Dieser Festkörperlaser besteht aus Neodym-dotiertem Yttrium-Aluminium-Granat und ist ein 4-Niveau-Laser mit einer Laser-Wellenlänge von $\lambda = 1064 \unit{nm}$. Er besitzt mehrere Absorptionslinien bei $804.4 \unit{nm}$, $808.4 \unit{nm}$, $812.9 \unit{nm}$ und $817.3 \unit{nm}$, die als Pump-Wellenlängen genutzt werden können.
Das Pumpen erfolgt durch einen Halbleiter-Diodenlaser. Durch Variation der Temperatur dehnt sich der Kristall aus und kann so auf verschiedene Wellenlängen um ca. $810 \unit{nm}$ eingestellt werden. Dieses Verhalten, die Ausgangsleistung und Laserschwelle werden zunächst charakterisiert.

Im Anschluss wird ein KTP-Kristall in den Resonator des Nd-YAG-Lasers eingebaut, sodass durch nichtlineare Effekte eine Frequenzverdopplung auf $532 \unit{nm}$ (grünes Licht) stattfindet.

Im letzten Versuchsteil wird der Laser in den Pulsbetrieb versetzt. Dafür wird anstelle des KTP-Kristalls eine Pockelszelle in den Resonator verbaut, die von einem Frequenzgenerator getrieben wird und zur aktiven Gütemodulation dient. Durch die Modulation werden deutlich höhere Peakleistungen erzielt. Diese werden mithilfe eines Photosensors vermessen.

\section{Charakterisierung des Diodenlasers}

\subsection{Messung des Absorptionsspektrums durch Variation der Temperatur}
\subsubsection{Aufbau und Durchführung}
\begin{figure}[htbp]
\centering
\includegraphics[width=\fullwidth]{aufbau_diodenlaser}
\caption{Versuchsaufbau zur Diodenlasers. Montiert sind Diodenlaser (A), Kollimator ($f = 6 \unit{mm}$, B), Fokussiereinheit ($f = 50 \unit{mm}$, C), Nd-YAG Stab (D) und Photodetektor (G). Zusätzlich ist der Diodenlaser mit einem Diodentreiber, auf dem sich die Diodentemperatur einstellen lässt, und der Photodetektor über einen Verstärker mit einem Oszilloskop verbunden. Bildquelle: \cite[S. 33]{anleitung1}}
\label{fig:aufbau_diodenlaser}
\end{figure}


Während der gesamten Versuchsreihe wird \emph{Versuchsaufbau 3} verwendet. 

Zuerst wird der Diodenlaser charakterisiert. Dazu wird im ersten Versuchsteil die Abhängigkeit zwischen geregelter Temperatur und der emittierten Wellenlänge betrachtet. Der Aufbau ist in Abbildung \ref{fig:aufbau_diodenlaser} gezeigt und besteht im wesentlichen aus dem Diodenlaser, dem Nd-YAG Stab und einem Photodetektor.
Die Temperatur kann in $0.1 \unit{\degree C}$ Schritten am Diodentreiber eingestellt werden. Zur Vermessung der Wellenlänge wird das bekannte Absorptionsspektrum von YAG verwendet. Die Intensität wird am Ende der optischen Bank mit einem Photodetektor vermessen, der mit einem Oszilloskop verbunden wird. Die Messwerte werden manuell vom Oszilloskop abgelesen und ihr Fehler aufgrund des Rauschens ebenfalls manuell abgeschätzt. Die Temperaturschritte sind arbiträr gewählt ca. $0.5 \unit{\degree C}$ zwischen den Minima und bis zu $0.1 \unit{\degree C}$ um die Minima.

Bei der Veränderung der Temperatur und damit der Wellenlänge werden mehrere Absorptionspeaks erwartet, die sich als Intensitätsminima messen lassen. 

\subsubsection{Messdaten}
\begin{figure}[htbp]
\centering
\includegraphics[width=\halfwidth]{temperatur}
\caption{Transmissionsspektrum von YAG. Die auf dem Oszilloskop abgelesene Spannung ist proportional zur Intensität und auf der vertikalen Achse aufgetragen, auf der horizontalen Achse befindet sich die Temperatur des Diodenlasers.}
\label{fig:temperatur}
\end{figure}

Die Messdaten sind in Abbildung \ref{fig:temperatur} zu sehen. Auf der vertikalen Achse wurde die auf dem Oszilloskop abgelesene Spannung mit den geschätzten Fehlern aufgetragen. Der Fehler auf die Temperatur beträgt $0.1 \unit{\degree C} / \sqrt{12} = 0.03 \unit{\degree C}$, da innerhalb eines Messwertes der Digitalanzeige eine Gleichverteilung angenommen wird.

\subsubsection{Auswertung}
Deutlich zu erkennen sind mehrere Intensitätsminima. Diese liegen bei ca. $25.5 \unit{\degree C}$, $29.0 \unit{\degree C}$, $32.3 \unit{\degree C}$, $34.8 \unit{\degree C}$, $37.5 \unit{\degree C}$ und $40.6 \unit{\degree C}$. Eine genaue Zuordnung zu den erwarteten Absorptionswellenlängen ist nicht nötig und möglich, da kein Model für den Zusammenhang bekannt ist, und da in den folgenden Versuchsteilen lediglich ein Absorptionsmaximum genutzt werden muss, welches in Form der Diodentemperatur angegeben wird.

\subsection{Bestimmung der Ausgangsleistung in Abhängigkeit des Diodenstroms}

\subsubsection{Aufbau und Durchführung}
\begin{figure}[htbp]
\centering
\includegraphics[width=\fullwidth]{aufbau_diodenlaser2}
\caption{Versuchsaufbau Bestimmung der Ausgangsleistung. Montiert sind Diodenlaser (A), Kollimator ($f = 6 \unit{mm}$, B), Fokussiereinheit ($f = 50 \unit{mm}$, C) und Leistungsmesser (I) mit 1000:1 Dämpfer. Bildquelle: \cite[S. 35]{anleitung1}}
\label{fig:aufbau_diodenlaser2}
\end{figure}

In diesem Versuchsteil wird die Ausgangsleistung in Abhängigkeit des Diodenstroms bei fester Temperatur vermessen. Es wird erwartet, dass ab einer bestimmten Laserschwelle die Laserleistung proportional zur Stromstärke ansteigt.

Der Versuchsaufbau ist in Abbildung \ref{fig:aufbau_diodenlaser2} abgebildet. Diesmal wird eine Temperatur von $38.6 \unit{\degree C}$ eingestellt, da sich nach Erwärmung des Lasers das fünfte Minimum bei dieser Temperatur befindet. Die gemessenen Intensitäten werden auf einem Leistungsmessgerät abgelesen, der eingestellte Messbereich beträgt zuerst $300 \unit{\upmu W}$, bis dieser überschritten ist und auf $1 \unit{mW}$ gewechselt wird. Aufgrund des 1000:1 Dämpfers müssen diese Werte jedoch mit dem Faktor 1000 multipliziert werden, um die Ausgangsleistung zu bestimmen. Die Stromstärke wird im Bereich von 0 bis 620 $\unit{mA}$ in $10 \unit{mA}$-Schritten eingestellt und je ein Leistungsmesswert notiert.

\subsubsection{Messdaten}
\begin{figure}[htbp]
\centering
\includegraphics[width=\halfwidth]{kennlinie_diode_raw}
\caption{Rohdaten zur Leistungsmessung des Diodenlasers. Auf der horizontalen Achse ist der am Treiber eingestellte Strom aufgetragen, die vertikale Achse zeigt die gemessene Leistung, die bereits mit dem Faktor 1000 multipliziert wurde.}
\label{fig:kennlinie_diode_raw}
\end{figure}

Abbildung \ref{fig:kennlinie_diode_raw} zeigt die Messdaten, dabei ist der eingestellte Strom auf der horizontalen Achse und die gemessene Leistung auf der vertikalen aufgetragen. Die Leistung ist bereits mit dem Faktor 1000 multipliziert, um für den 1000:1 Dämpfer zu kompensieren. Der Fehler auf die Stromeinstellung beträgt $1 \unit{mA} / \sqrt{12} = 0.3 \unit{mA}$, da der Diodentreiber über eine Digitalanzeige verfügt, auf der sich nur Milliampere einstellen lassen. Der Fehler auf die gemessene Leistung beträgt analog $0.3 \unit{mW}$, da auch hier eine Digitalanzeige verwendet wurde. Da die Fehlerbalken in dieser Auftragung sehr klein sind, werden sie auf dem Plot als Kreuze abgebildet.

\subsubsection{Auswertung}
Das Kernstück der Analyse ist eine Geradenanpassung. Dafür muss zuerst eine Offsetkorrektur durchgeführt werden und ein valider Bereich für den Fit ausgewählt werden. 

Für die Bestimmung des Offsets wird der Mittelwert aller gemessenen Leistungen mit $I < 150 \unit{mA}$ gebildet. Dieser beträgt $-1.64 \unit{mW}$ und wird nun von der Leistung subtrahiert, sodass diese im unteren Bereich $0 \unit{mW}$ beträgt.

Der valide Anpassungsbereich wird manuell geschätzt. Es werden zur Anpassung ausschließlich Werte $I > 200 \unit{mA}$ genutzt, da die Laserschwelle augenscheinlich darunter liegt.

\begin{figure}[htbp]
\centering
\includegraphics[width=\halfwidth]{kennlinie_diode_0_fit}
\includegraphics[width=\halfwidth]{kennlinie_diode_0_residual}
\caption{Geradenanpassung im Bereich $I > 200 \unit{mA}$. Der resultierende Threshold liegt bei $(188.10 \pm 0.15) \unit{mA}$. Im rechten Teil sind die Residuen gezeigt, $\chi^2 / \textrm{ndf} = 40$, außerdem zeigt sich eine deutliche Systematik.}
\label{fig:kennlinie_diode_0_fit}
\end{figure}

Zur Anpassung wird der Migrad-Algorithmus verwendet. Das Ergebnis der Anpassung ist in Abbildung \ref{fig:kennlinie_diode_0_fit} gezeigt. Der Threshold beträgt demnach $(188.10 \pm 0.15) \unit{mA}$, und die Leistungssteigerung $(0.8530 \pm 0.0005) \unit{W/A}$. Allerdings lässt der Residuenplot (rechte Seite der Abbildung \ref{fig:kennlinie_diode_0_fit}) erkennen, dass ein systematischer Fehler vorliegt und das Model separat an die Daten von $200 \unit{mA}$ bis $410 \unit{mA}$ und über $410 \unit{mA}$ angepasst werden muss.

\begin{figure}[htbp]
\centering
\includegraphics[width=\halfwidth]{kennlinie_diode_1_fit}
\includegraphics[width=\halfwidth]{kennlinie_diode_1_residual}
\includegraphics[width=\halfwidth]{kennlinie_diode_2_fit}
\includegraphics[width=\halfwidth]{kennlinie_diode_2_residual}
\caption{Geradenanpassung im Bereich von 200 bis 410 $\unit{mA}$ (oben), sowie im Bereich $> 410 \unit{mA}$ (unten). Die Residuengraphen zeigen eine deutlich bessere Anpassung, es sind nur noch Systematiken durch Binning zu erkennen.}
\label{fig:kennlinie_diode_12_fit}
\end{figure}

\begin{table}[htbp]
\centering
\begin{tabular}{|c||c|c|c|}
\hline
Strombereich & Threshold & Slope & $\chi^2 / \textrm{ndf}$ \\ \hline \hline
alles $ > 200 \unit{mA}$ & $(188.10 \pm 0.15) \unit{mA}$ & $(0.8530 \pm 0.0005) \unit{W/A}$ & $40.0$ \\
200 bis 410 $\unit{mA}$ & $(182.10 \pm 0.23) \unit{mA}$ & $(0.8139 \pm 0.0013) \unit{W/A}$ & $2.1$ \\
$ > 410 \unit{mA}$ & $(201.6 \pm 0.5) \unit{mA}$ & $(0.8892 \pm 0.0014) \unit{W/A}$ & $2.2$ \\
\hline
\end{tabular}
\caption{Ergebnisse von Anpassungen für verschiedene Bereiche}
\label{tbl:diode}
\end{table}

Das Ergebnis dieser eingeschränkten Anpassung ist in Abbildung \ref{fig:kennlinie_diode_12_fit} und Tabelle \ref{tbl:diode} gezeigt. Die Anpassungen sind deutlich besser gelungen und weisen nur noch eine Binning-Systematik auf. 

Möglicher Grund für die Separation der Daten in zwei Bereiche ist eine Veränderung des Messaufbaus, sowie das Umschalten des Messbereiches des Leistungsmessers. 

Da der untere Bereich aufgrund seiner Nähe zum Laserthreshold den geringeren Fehler aufweist, wird dieses Ergebnis als finales Ergebnis genutzt:
Der Laserthreshold der Laserdiode beträgt 
\[ (182.10 \pm 0.23) \unit{mA} \]
Darüber hinaus beträgt die Leistungssteigerung bei Stromänderung
\[ (0.8139 \pm 0.0013) \unit{W/A} \]


\section{Charakterisierung des Nd-YAG-Lasers}

\subsection{Lebensdauer des $^4F_{3/2}$-Niveaus}
\subsubsection{Aufbau und Durchführung}
\begin{figure}[htbp]
\centering
\includegraphics[width=\fullwidth]{aufbau_lebensdauer}
\caption{Versuchsaufbau zur Bestimmung der Lebensdauer des $^4F_{3/2}$-Niveaus. Montiert sind Diodenlaser (A), Kollimator ($f = 6 \unit{mm}$, B), Fokussiereinheit ($f = 50 \unit{mm}$, C), Nd-YAG Stab (D), Filter RG 1000 (Transmission nur $\lambda > 1000 \unit{nm}$, F) und Photodetektor (G). Letzterer ist über einen Verstärker an ein Oszilloskop angeschlossen. Bildquelle: \cite[S. 35]{anleitung1}}
\label{fig:aufbau_lebensdauer}
\end{figure}

In diesem Versuchsteil wird der Nd-YAG-Stab erneut eingebaut und vermessen. Der Diodenlaser wird im Pulsbetrieb betrieben. Es wird erwartet, dass die Infrarotintensität des Neodyms der Schaltung des Pumplasers träge folgt und daher nach Ein- und Abschalten ein exponentieller Anstieg bzw. Abfall zu messen ist.

Der Versuchsaufbau befindet sich in Abbilung \ref{fig:aufbau_lebensdauer}. Die Laserdiode wird im Pulsbetrieb bei ca. $160 \unit{Hz}$ und mit einem Strom von $197 \unit{mA}$, also knapp über der Laserschwelle, betrieben. Erneut beträgt die eingestellte Temperatur $38.6 \unit{\degree C}$.

Die Daten werden am Oszilloskop abgelesen und zusätzlich auf einem USB-Stick als \texttt{csv}-Datei gespeichert, sodass sie separat analysiert werden können. Ein manuelles Ablesen der Lebensdauer mithilfe der Oszilloskopcursor ergibt $240 \unit{\mu s}$. 

\subsubsection{Messdaten}
\begin{figure}[htbp]
\centering
\includegraphics[width=\halfwidth]{F0010TEK}
\caption{Rohdaten auf dem Oszilloskop. Eine Schnelleinschätzung von $\tau = 240 \unit{\upmu s}$ ist zu sehen.}
\label{fig:lebensdauer_oszi}
\end{figure}

Die Rohdaten zu diesem Versuchsteil (Abbildung \ref{fig:lebensdauer_oszi}) entstammen direkt dem Oszilloskop. Die obere Linie entspricht die Messdaten von Kanal 1, der direkt mit dem Diodentreiber verbunden ist und als Trigger dient. Dieser Kanal ist invertiert, sodass das höhere Spannungsniveau zu einer ausgeschalteten Diode korrespondiert. Der Zeitpunkt des Abschaltens des Diodenlasers ist mit dem Cursor markiert und wird im folgenden als $t = 0$ festgelegt.

Auf Kanal 2 sind die Messdaten des Photodetektors zu sehen. Es zeigt sich, dass bei $t < 0$ eine gesättigte Intensität gemessen wird, nach Abschalten der Diode fällt die gemessene Intensität jedoch wie erwartet exponentiell auf einen geringeren Wert ab.

\subsubsection{Auswertung}
Um die Lebensdauer genauer zu bestimmen, wird eine Anpassung einer Exponentialfunktion durchgeführt. Dafür werden zunächst die Fehler auf die Spannungsmesswerte geschätzt. Es bietet sich dabei an, den Stichprobenfehler der Daten mit $t < 0$ zu berechnen. Dieser beträgt $\sigma = 2.75 \unit{mV}$ und wird im folgenden als Fehler auf alle Spannungswerte genutzt.

Nun wird eine Anpassung an eine Funktion der Form 
\[ f(t) = A \exp\left(-\frac{t}{T}\right) + \textrm{offset} \] durchgeführt. Dadurch bezeichnet $T$ die Lebensdauer des $^4F_{3/2}$-Niveaus. Die anderen Fitparameter sind nicht von Interesse.

\begin{figure}[htbp]
\centering
\includegraphics[width=\halfwidth]{yag_lifetime}
\caption{Ausgewertete Kurve des Oszilloskops. Auf die Spannungswerte wird ein Fehler von $2.75 \unit{mV}$ angenommen, die Qualität der Anpassung ist $\chi^2 / \textrm{ndf} = 1.01$}
\label{fig:lebensdauer_fit}
\end{figure}

Das Ergebnis der Anpassung ist in Abbildung \ref{fig:lebensdauer_fit} gezeigt. Die Anpassung erfolgt mit $\chi^2 / \textrm{ndf} = 751.8 / 746 = 1.01$. Für die Lebensdauer ergibt sich
\[
	T = (250.7 \pm 2.5) \unit{\upmu s}
\]
Dieser Wert ist $4 \sigma$ von der Grobabschätzung mit den Oszilloskopcursorn ($240 \unit{\upmu s}$) und $8 \sigma$ vom Literaturwert $230 \unit{\upmu s}$ entfernt.

\subsection{Bestimmung der Ausgangsleistung in Abhängigkeit der Pumpleistung}
\subsubsection{Aufbau und Durchführung}
\begin{figure}[htbp]
\centering
\includegraphics[width=\fullwidth]{aufbau_ndyag}
\caption{Versuchsaufbau zur Messung der Ausgansleistung. Montiert sind Diodenlaser (A), Kollimator ($f = 6 \unit{mm}$, B), Fokussiereinheit ($f = 50 \unit{mm}$, C), Nd-YAG Stab (D), Spiegel $100-2 \unit{\%}$ (E), Filter RG 1000 (Transmission nur $\lambda > 1000 \unit{nm}$, F) und Leistungsmesser (I) mit 1000:1 Dämpfer. Bildquelle: \cite[S. 36]{anleitung1}}
\label{fig:aufbau_ndyag}
\end{figure}

In diesem Versuchsteil wird zum ersten mal der Resonator des Nd-YAG-Festkörperlasers aufgebaut (siehe Abbildung \ref{fig:aufbau_ndyag}). Dieser besteht aus einem auf den YAG-Stab aufgedampften Spiegel und einem einzelnen $100 - 2 \unit{\%}$ Spiegel. Aus letzterem wird die infrarote Laserstrahlung der Wellenlänge $\lambda = 1064 \unit{nm}$ emittiert. Da dort ebenfalls Anteile des Pumplichtes austreten, werden diese vor dem Photodetektor durch einen RG 1000 Filter absorbiert.

Die Leistung dieses Lasers wird in Abhängigkeit des Diodenstromes gemessen, mithilfe der vorherigen Charakterisierung der Diode kann so die Lasereffizienz berechnet werden.

Der Diodenstrom wird in $10 \unit{mA}$-Schritten von 0 bis 620 $\unit{mA}$ variiert. Als Diodentemperatur werden $38.5 \unit{\degree C}$ eingestellt.
Die Intensität wird am Leistungsmesser auf einer Digitalanzeige abgelesen, ihr Fehler wird über die Schwankungen manuell geschätzt und beträgt je nach Messwert zwischen $3 \unit{\upmu W}$ und $100 \unit{\upmu W}$.

\subsubsection{Messdaten}
\begin{figure}[htbp]
\centering
\includegraphics[width=\halfwidth]{kennlinie_yag_raw}
\includegraphics[width=\halfwidth]{kennlinie_yag_raw2}
\caption{Rohdaten zur Leistungsmessung des Festkörperlasers. Links: Auf der horizontalen Achse ist der am Diodentreiber eingestellte Strom aufgetragen, die vertikale Achse zeigt die gemessene Leistung, die bereits mit dem Faktor 1000 multipliziert wurde. Rechts: Die horizontale Achse zeigt die Diodenleistung, die mithilfe einer abschnittsweise definierten Funktion berechnet wurde.}
\label{fig:kennlinie_yag_raw}
\end{figure}

In den Messdaten (Abbildung \ref{fig:kennlinie_yag_raw}, linke Seite) zeigt sich das erwartete Muster: die ersten Werte unterhalb der Laserschwellen sind ungefähr $0 \unit{mW}$, über der Schwelle steigt die Leistung monoton an. 

\subsubsection{Auswertung}
Zuerst wird erneut eine Offsetkorrektur durchgeführt. Dafür wird von allen Daten ein Offset von $0.02 \unit{mW}$ subtrahiert, da dies der Mittelwert der Daten für $I < 150 \unit{mA}$ ist.

Danach werden die Daten in Abhängigkeit der Diodenleistung aufgetragen. Dazu werden die Erkenntnisse des Vorversuches (Tabelle \ref{tbl:diode}) genutzt, um eine abschnittsweise definierte Funktion zu bilden, die diese Abhängigkeit beschreibt:
\[
	P(I) = \left\{ \begin{array}{ll}	0 & I < 182.10 \unit{mA} \\
									(I - (182.10 \pm 0.23) \unit{mA}) \cdot (0.8139 \pm 0.0013) \unit{W/A} & 182.10 \unit{mA} < I < 410 \unit{mA} \\
									(I - (201.6 \pm 0.5) \unit{mA}) \cdot (0.8892 \pm 0.0014) \unit{W/A} & I > 410 \unit{mA}\end{array} \right.
\]

Mithilfe dieser Funktion und Gaußscher Fehlerfortpflanzung kann nun die Diodenleistung auf der horizontalen Achse aufgetragen werden (Abbildung \ref{fig:kennlinie_yag_raw}, rechte Seite).

\begin{figure}
\centering
\includegraphics[width=\halfwidth]{kennlinie_yag_linear_fit}
\includegraphics[width=\halfwidth]{kennlinie_yag_linear_residual}
\caption{Lineare Anpassung an die Leistungsmessung. Diese schlägt fehl, $\chi^2 / \textrm{ndf} = 410$. Die Residuen zeigen, dass ein Polynom höheren Gerades vermutlich besser angepasst werden kann.}
\label{fig:kennlinie_yag_linear}
\end{figure}

In Abbildung \ref{fig:kennlinie_yag_linear} ist eine lineare Anpassung an die Leistungsmessung gezeigt. Wie $\chi^2 / \textrm{ndf} = 410$ zeigt, ist diese Anpassung ungültig (oder sehr unwahrscheinlich). Daher wird im Folgenden ein Polynom zweiten Grades angepasst.

\begin{figure}
\centering
\includegraphics[width=\halfwidth]{kennlinie_yag_0_fit}
\includegraphics[width=\halfwidth]{kennlinie_yag_0_residual}
\caption{Anpassung eines Polynoms zweiten Gerades an die Leistungsmessung. Daraus berechnet sich eine Schwellleistung von $(16.68 \pm 0.68) \unit{mW}$. Die Effizienz $\frac{dP_\textrm{out}}{dP_\textrm{in}}$ beträgt bei beispielsweise $200 \unit{mW}$ Eingangsleistung $(2.280 \pm 0.018) \%$.}
\label{fig:kennlinie_yag_0}
\end{figure}

Eine solche Anpassung ist in Abbildung \ref{fig:kennlinie_yag_0} zu sehen. Aus den Vorfaktoren im Polynom kann nun die Nullstelle berechnet werden, die die Laserschwelle darstellt. Daraus ergibt sich eine Schwellleistung von
\[
	(16.68 \pm 0.34) \unit{mW}
\]

\begin{figure}
\centering
\includegraphics[width=\halfwidth]{kennlinie_yag_0_efficiency}
\caption{Abhängigkeit der Effizienz $\frac{dP_\textrm{out}}{dP_\textrm{in}}$ von der Pumpleistung.}
\label{fig:kennlinie_yag_0_efficiency}
\end{figure}

Die Effizienz des Vorgangs hängt aufgrund der quadratischen Anpassung linear von der Pumpleistung ab. Der Zusammenhang ist in Abbildung \ref{fig:kennlinie_yag_0_efficiency} abgebildet. Demnach beträgt sie zwischen 1.5 und 3 $\unit{\%}$.

\subsection{Spiking}
\subsubsection{Aufbau und Durchführung}
In diesem Versuchsteil wird Spiking betrachtet. Beim Einschalten des Pumplasers entsteht zuerst eine Leistungsspitze, die in einer anharmonischen, gedämpften Schwingung auf die Laserleistung abfällt. 
Der Versuchsaufbau ist analog zum letzten Versuchsteil (Abbildung \ref{fig:aufbau_ndyag}), außer dass anstelle des Leistungsmessers erneut der Photodetektor verwendet wird, der an ein Oszilloskop angeschlossen ist. Der Diodenlaser wird moduliert mit ca $4.75 \unit{kHz}$. 
Ebenfalls an das Oszilloskop angeschlossen ist ein Signal, dass proportional zum Pumpstrom ist, sodass auf das modulierte Signal getriggert werden kann.

\subsubsection{Messdaten}
\begin{figure}
\centering
\includegraphics[width=\halfwidth]{F0015TEK}
\includegraphics[width=\halfwidth]{spiking}
\caption{Rohdaten des Spikings. Linke Seite: direktes Bildschirmfoto vom Oszilloskop. Getriggert wurde auf die Modulationsfrequenz, die mit $4.75 \unit{kHz}$ angegeben ist. Rechte Seite: aus gespeicherten Daten rekonstruierter Verlauf. Die Zeitskala wurde angepasst, sodass der Beginn des Spikings bei $t = 0$ abgebildet ist. Außerdem wurde eine Offsetkorrektur mit den Daten von $t < 0$ durchgeführt.}
\label{fig:spiking}
\end{figure}

In Abbildung \ref{fig:spiking} sind auf der linken Seite die Rohdaten vom Oszilloskop zu sehen. Auf der rechten Seite sind die aus der \texttt{csv}-Datei rekonstruierten Daten abgebildet, die Zeitskala ist so verschoben, dass der Spiking-Prozess bei $t = 0$ beginnt. Auf der vertikalen Achse ist die gemessene Spannung aufgetragen, die proportional zur Intensität ist. Sie ist so korrigiert, dass bei $t < 0$ die Spannung im Mittel bei $U = 0$ liegt.
Diese Daten werden in der Auswertung nur qualitativ betrachtet.

\subsubsection{Auswertung}
Es ist deutlich zu erkennen, dass die Spannung zuerst auf ca. $4.0 \unit{V}$ springt und sich danach anharmonisch und gedämpft auf $2.0 \unit{V}$ einschwingt. Der gesamte Prozess dauert ca. $60 \unit{\upmu s}$.

\section{Frequenzverdopplung mit KTP}
\subsection{Aufbau und Durchführung}
\begin{figure}[htbp]
\centering
\includegraphics[width=\fullwidth]{aufbau_ktp}
\caption{Versuchsaufbau zur Frequenzverdopplung. Montiert sind Diodenlaser (A), Kollimator ($f = 6 \unit{mm}$, B), Fokussiereinheit ($f = 50 \unit{mm}$, C), Nd-YAG Stab (D), KTP-Kristall (H), Spiegel $100-2 \unit{\%}$ (E), Filter BG 39 (Transmission nur $\lambda \approx 532 \unit{nm}$, F) und Leistungsmesser (I) (\emph{ohne} 1000:1 Dämpfer). Bildquelle: \cite[S. 38]{anleitung1}}
\label{fig:aufbau_ktp}
\end{figure}

In diesem Versuchsteil wird ein KTP-Kristall in den Laserresonator eingebaut. Aufgrund nichtlinearer optischer Effekte wird die Frequenz der Laserstrahlung verdoppelt, die Wellenlänge halbiert sich dadurch von $1064 \unit{nm}$ auf $532 \unit{nm}$. Das emittierte Licht ist damit für das menschliche Auge sichtbar und erscheint grün.

Der Versuchsaufbau ist in Abbildung \ref{fig:aufbau_ktp} zu sehen. Wie im vorherigen Versuch ist der Festkörperlaser vollständig aufgebaut, jedoch ist in den Resonator ein KTP-Kristall eingebaut. Die Position der Laserspiegel ist dementsprechend korrigiert, sodass nun grünliches Licht am Austrittsspiegel emittiert wird und den Filter passiert. Die Leistung wird erneut mit dem Leistungsmessgerät vermessen, diesmal ohne 1000:1 Dämpfer.


\section{Aktive Gütemodulation}

\section{Zusammenfassung}

\newpage
\begin{thebibliography}{xxxx}
\bibitem{anleitung1} Praktikumsanleitung:
Versuch F08, Neodym-YAG-Laser, Teil 1. URL: \url{http://institut2a.physik.rwth-aachen.de/de/teaching/praktikum/Anleitungen/B_FK08-Nd-YAG_Teil1_12-03-2012.pdf} [Stand: 20.03.2015]
\bibitem{anleitung2} Praktikumsanleitung:
Versuch F08, Neodym-YAG-Laser, Teil 2. URL: \url{http://institut2a.physik.rwth-aachen.de/de/teaching/praktikum/Anleitungen/B_FK08-Nd-YAG-Teil2_18-01-2012.pdf} [Stand: 20.03.2015]
\end{thebibliography}


\end{document}
