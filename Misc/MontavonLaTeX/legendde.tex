\noindent Im Folgenden wird die Notationskonvention, die den Texten und mathematischen Ausführungen zu Grunde liegt, erläutert, damit der Leser diese gegebenenfalls nachschlagen kann. \\ \quad \\ 
{\bf Funktionen und Variablen}
\begin{itemize}
\item Bekannte Standardfunktionen, beispielsweise $\sin \left( x \right) $, werden aufrecht notiert. Dies gilt auch f\"ur die Exponentialfunktion, die als $\expe^x$ oder $\exp \left( x \right)$ geschrieben wird.
\item Ebenfalls aufrecht wird die imagin\"are Einheit $\im$ notiert. 
\item Anderweitig definierte Funktionen, insbesondere Polynome, werden kursiv notiert und sind an ihrer Abh\"angigkeit zu erkennen, zum Beispiel $ f\left( x \right)$. 
\item Variablen und Parameter sind grunds\"atzlich kursiv notiert. 
\end{itemize}
{\bf Operationen}
\begin{itemize}
\item Differentialformen werden mit aufrechtem $\di$ geschrieben, beispielsweise $\diff{f}{x}$. 
\item Die partielle Ableitung wird ausgeschrieben, zum Beispiel $\pdiff{f}{x}$
\item Der zeitliche Mittelwert einer Gr\"oße wird durch $\overline f = \langle f \left( t \right) \rangle$ repr\"asentiert.  
\item Punkte \"uber Funktionsnamen bezeichnen die zeitliche Ableitung, also $\dot f \left( t \right) = \diff{f}{t} $. 
\item Der Nablaoperator dient zur Darstellung der Operationen Gradient $\vecb \nabla$, Divergenz $\vecb \nabla \cdot $, Rotation $\vecb \nabla \times$ sowie des Laplace-Operators $\vecb \Delta$. 
\end{itemize}
{\bf Mehrdimensionale Objekte}
\begin{itemize}
\item Vektoren werden fett und kursiv gedruckt dargestellt, beispielsweise $\vecb v$
\item Matrizen sind fett und aufrecht gedruckt, so wie $ \matb A$. 
\item Das Skalarprodukt wird mit einem Punkt dargestellt, also $\vecb a \cdot \vecb b$.
\item Der Betrag eines Vektors wird mit einfachen Strichen dargestellt, zum Beispiel $\Norm{ \vecb v}$.
\item Zu einem Vektor $\vecb k $ bezeichnet $\vecb{\hat e}_k = \frac{\vecb k}{\Norm{\vecb k}} $ den zugeh\"origen Einheitsvektor.
\item $\nullvec$ ist der Nullvektor.
\item Auf andere Tensoren wird gesondert hingewiesen.  
\end{itemize}
{\bf Verwendete Koordinatensysteme}
\begin{itemize}
\item Ohne weiteren Hinweis werden kartesische Koordinaten $\left( x,y,z \right)$ mit den Einheitsvektoren $\vecbe x, \vecbe y, \vecbe z$ verwendet.
\item Zylinderkoordinaten werden als $\left( \rho, \varphi, z\right) $ mit den Einheitsvektoren $\vecbe \rho, \vecbe \varphi , \vecbe z$ notiert.
\item Kugelkoordinaten werden durch $\left( r, \varphi, \vartheta \right)$ mit den Einheitsvektoren $\vecbe r, \vecbe \varphi, \vecbe \vartheta$ ausgedr\"uckt. 
\end{itemize}
{\bf Einheiten und Zehnerpotenzen}
\begin{itemize}
\item Als Trennzeichen wird der Punkt verwendet, also $\frac 1 2 = 0.5$.
\item Einheiten und deren metrischen Vorsilben werden mit Abstand und aufrecht geschrieben, beispielsweise $ 9.81 \unit{\frac m{s^2}} $.
\item Zehnerpotenzen werden deutlich angehangen und nicht abgek\"urzt, zum Beispiel $4.3 \pow{-7} $
\end{itemize}
{\bf Textelemente}
\begin{itemize}
\item Zitate werden wie folgt angegeben: \guillemotright Da steh ich nun, ich armer Tor, und bin so klug als wie zuvor!\guillemotleft
\item Fußnoten sind am Ende der Seite aufgef\"uhrt, es existieren keine Endnoten.
\end{itemize}
